%%%%%%%%%%%%%%%%%%%%%%%%%%%%%%%%%%%%%%%%%
% McMaster Masters/Doctoral Thesis 
% LaTeX Template
% Version 2.2 (11/23/15)
%
% This template has been downloaded from:
% http://www.LaTeXTemplates.com
% Then subsequently from http://www.overleaf.com
%
% Version 2.0 major modifications by:
% Vel (vel@latextemplates.com)
%
% Original authors:
% Steven Gunn  (http://users.ecs.soton.ac.uk/srg/softwaretools/document/templates/)
% Sunil Patel (http://www.sunilpatel.co.uk/thesis-template/)
%
% Modified to McMaster format by Benjamin Furman (http://www.benjaminfurman.wordpress.com
%
% License:
% CC BY-NC-SA 3.0 (http://creativecommons.org/licenses/by-nc-sa/3.0/)
%
%%%%%%%%%%%%%%%%%%%%%%%%%%%%%%%%%%%%%%%%%

%----------------------------------------------------------------------------------------
% DOCUMENT CONFIGURATIONS
%----------------------------------------------------------------------------------------

\documentclass[
11pt, % The default document font size, options: 10pt, 11pt, 12pt
oneside, % Two side (alternating margins) for binding by default, uncomment to switch to one side
english, % ngerman for German
%singlespacing, % Single line spacing, alternatives: onehalfspacing or doublespacing
doublespacing
%draft, % Uncomment to enable draft mode (no pictures, no links, overfull hboxes indicated)
%nolistspacing, % If the document is onehalfspacing or doublespacing, uncomment this to set spacing in lists to single
%liststotoc, % Uncomment to add the list of figures/tables/etc to the table of contents
%toctotoc, % Uncomment to add the main table of contents to the table of contents
%parskip, % Uncomment to add sp ace between paragraphs
]{McMasterThesis} % The class file specifying the document structure

%----------------------------------------------------------------------------------------
% Import packages here
%----------------------------------------------------------------------------------------
\usepackage[utf8]{inputenc} % Required for inputting international characters
\usepackage[T1]{fontenc} % Output font encoding for international characters
\usepackage{graphicx}
\usepackage{palatino} % Use the Palatino font by default
\usepackage{lastpage} % count pages
\usepackage{siunitx} % for scientific units (micro-liter, etc)
\usepackage{rotating}

\DeclareGraphicsExtensions{.pdf,.png,.jpg,.jpeg}

%----------------------------------------------------------------------------------------
% Handling Citations
%----------------------------------------------------------------------------------------
\usepackage[backend=bibtex,firstinits=true,style=authoryear,natbib=true,sorting=nyt,maxbibnames=99]{biblatex} % User the bibtex backend with the authoryear citation style (which resembles APA)
% can change the maxbibnames to cut long author lists to specified length followed by et al., currently set to 99. 
\addbibresource{bibliography.bib} % The filename of the bibliography
\usepackage[autostyle=true]{csquotes} % Required to generate language-dependent quotes in the bibliography

\DeclareNameAlias{sortname}{last-first}
%----------------------------------------------------------------------------------------
% Collect all your header information from the chapters here, each header or the chapters should be pretty clean
%----------------------------------------------------------------------------------------
\usepackage{parskip} %this will put spaces between my paragraphs
\setlength{\parindent}{15pt} % this will create and indent on all but the first paragraph of each section. 

%\usepackage{acronym}
%\acrodef{est}[EST]{expressed sequence tag}
%\acrodef{Xl}[\textit{X.\,laevis}]{\textit{Xenopus laevis}}
\usepackage{acro}
\usepackage{amsmath}
\usepackage{subcaption}
\usepackage{lineno}
%\linenumbers

\DeclareAcronym{EC}{
	short = EC,
	long = entorhinal cortex
}
\DeclareAcronym{DG}{
	short = DG,
 	long = dentate gyrus
}
\DeclareAcronym{DGC}{
	short = DGC,
 	long = dentate granule cell
}
\DeclareAcronym{NPC}{
	short = NPC,
	long = neural progenitor cell
}
\DeclareAcronym{AHN}{
	short = AHN,
	long = adult hippocampal neurogenesis
}
\DeclareAcronym{PP}{
	short = PP,
	long = perforant pathway
}
\DeclareAcronym{TA}{
	short = TA,
	long = temporoammonic pathway
}
\DeclareAcronym{ANN}{
	short = ANN,
	long = artificial neural network
}
\DeclareAcronym{FFNN}{
	short = FFNN,
	long = feed forward neural network
}
\DeclareAcronym{RNN}{
	short = RNN,
	long = recurrent neural network
}
\DeclareAcronym{LSTM}{
	short = LSTM,
	long = long short-term memory
}
\DeclareAcronym{ESN}{
	short = ESN,
	long = echo state network
}
\DeclareAcronym{LSM}{
	short = LSM,
	long = liquid state machine
}
\DeclareAcronym{RBM}{
	short = RBM,
   	long = restricted boltzmann machine
}
\DeclareAcronym{CD}{
	short = CD,
	long = contrastive divergence
}
\DeclareAcronym{CRBM}{
	short = CRBM,
 	long = conditional restricted boltzmann machine
}
\DeclareAcronym{TRBM}{
	short = TRBM,
	long = temporal restricted boltzmann machine
}
\DeclareAcronym{DBN}{
	short = DBN,
	long = deep belief network
}
\DeclareAcronym{DNC}{
	short = DNC,
	long = dynamic node creation
}

\usepackage{etoolbox}
\preto\chapter{\acresetall} % resets acronyms for each chapter
%\preto\section{\acresetall} % reset acronyms after each section

\usepackage{xspace} %helps spacing with custom commands. 
\usepackage{cleveref}

\usepackage{pgfplotstable} % a much better way to handle tables
% can drop in table information here, then call later, if you choose (as opposed to calling a file or typing in/copy--paste by hand). 
% \usepackage{float} % if you need to demand figure/table placement, then this will allow you to use [H], which should be strong. 

%----------------------------------------------------------------------------------------
%	THESIS INFORMATION
%----------------------------------------------------------------------------------------

\thesistitle{Computational Modelling of Adult Hippocampal Neurogenesis} % Your thesis title, print it elsewhere with \ttitle
\supervisor{Dr. Suzanna Becker} % Your supervisor's name, print it elsewhere with \supname
\examiner{} % Your examiner's name, print it elsewhere with \examname
\degree{Master of Science} % Your degree name, print it elsewhere with \degreename
\author{Rory Finnegan} % Your name, print it elsewhere with \authorname
\addresses{102 Gladstone Ave., Hamilton, Ontario, L8M 2H9, Canada} % Your address, print it elsewhere with \addressname

\subject{Computational Neuroscience} % Your subject area, print it elsewhere with \subjectname
\keywords{Hippocampus, Neurogenesis, Modelling, Neuroscience, Boltzmann, Computational} % Keywords for your thesis, print it elsewhere with \keywordnames
\university{\href{http://www.mcmaster.ca/}{McMaster University}} % Your university's name and URL, print it elsewhere with \univname
\department{\href{http://neurosciencemcmaster.ca/}{McMaster Integrative Neuroscience \& Discovery}} % Your department's name and URL, print it elsewhere with \deptname
\group{\href{http://graduate.mcmaster.ca/pnb/department/becker/index.html}{Neurotechnology and Neuroplasticity Lab}} % Your research group's name and URL, print it elsewhere with \groupname
\faculty{\href{http://www.science.mcmaster.ca/}{Faculty of Science}} % Your faculty's name and URL, print it elsewhere with \facname

\hypersetup{pdftitle=\ttitle} % Set the PDF's title to your title
\hypersetup{pdfauthor=\authorname} % Set the PDF's author to your name
\hypersetup{pdfkeywords=\keywordnames} % Set the PDF's keywords to your keywords

\begin{document}

 \frontmatter % Use roman page numbering style (i, ii, iii, iv...) for the pre-content pages

\pagestyle{plain} % Default to the plain heading style until the thesis style is called for the body content

%----------------------------------------------------------------------------------------
%	Half Title (lay title)
%----------------------------------------------------------------------------------------
%\begin{halftitle} % could not get this environment working
%\vspace*{\fill}
\vspace{6cm}
\begin{center}
Computational Modelling of Adult Hippocampal Neurogenesis % ideally, but it doesn't matter
\end{center}
%\vspace*{\fill}
\pagenumbering{gobble} % leave this here, McMaster doesn't want this page numbered
%\end{halftitle}
\clearpage

%----------------------------------------------------------------------------------------
%	TITLE PAGE
%----------------------------------------------------------------------------------------
% not exactly the Grad Studies example, but much nicer. 
\pagenumbering{gobble}
%\begin{titlepage} % had to turn off to get page numbering right
\begin{center}

\vfill
\textsc{\Large \ttitle} \\

\vfill
By \authorname, \\%% -----> List degrees after comma  <----


%% nicer title page
%\textsc{\LARGE \univname}\\[1.5cm] % University name
%\textsc{\Large Doctoral Thesis}\\[0.5cm] % Thesis type

%\HRule \\[0.4cm] % Horizontal line
%{\huge \bfseries \ttitle}\\[0.4cm] % original overleaf style
%{\huge \sc \ttitle}\\[0.4cm] % Thesis title
%\HRule \\[1.5cm] % Horizontal line
 
%\begin{minipage}{0.4\textwidth}
%\begin{flushleft} \large
%\emph{Author:}\\
%\href{http://www.johnsmith.com}{\authorname \\ M.Sc.} % Author name - remove the \href bracket to remove the link
%\end{flushleft}
%\end{minipage}
%\begin{minipage}{0.4\textwidth}
%\begin{flushright} \large
%\emph{Supervisor:} \\
%\href{http://www.jamessmith.com}{\supname} % Supervisor name - remove the \href bracket to remove the link  
%\end{flushright}
%\end{minipage}\\[3cm]
 \vfill
{\large \textit{A Thesis Submitted to the School of Graduate Studies in the Partial Fulfilment of the Requirements for the Degree \degreename}}\\
%\\[0.3cm] % University requirement text
%\textit{in the}\\[0.4cm]
% \groupname\\
%\deptname\\[2cm] % Research group name and department name


\vfill

%Insert Creative Commons Artwork
\DeclareGraphicsExtensions{.pdf,.png,.jpg}
\begin{center}
{\large \univname\, \today}\\[4cm] 
\leavevmode
%Insert image file name below "cc-by-nc-nd.png"
\includegraphics[width=1in]{Figs/cc-by-4.png}
\end{center}
\label{fig:cc}
%insert a link to the licence and its description below
\scriptsize{This thesis is licensed under a \href{http://creativecommons.org/licenses/by/4.0/}{Creative Commons Attribution 4.0 International License.}}

%{\large \univname\, \copyright\, Copyright by \authorname\, \today}\\[4cm] % Date: replace \today with \date{} <---input date


%\vfill
\end{center}


%\end{titlepage}


%----------------------------------------------------------------------------------------
%	Descriptive note numbered ii
%----------------------------------------------------------------------------------------
% Need to add below info
\newpage
\pagenumbering{roman} % leave to turn numbering back on
\setcounter{page}{2} % leave here to make this page numbered ii, a Grad School requirement

\noindent % stops indent on next line
\univname \\ 
\degreename\, (\the\year) \\
Hamilton, Ontario (\deptname) \\[1.5cm]
TITLE: \ttitle \\
AUTHOR: \authorname\,  %list previous degrees
(\univname)  \\
SUPERVISOR: \supname\, \\ 
NUMBER OF PAGES: \pageref{lastoffront}, \pageref{LastPage}  % put in iv and number

\clearpage

%----------------------------------------------------------------------------------------
%	Lay abstract number iii
%----------------------------------------------------------------------------------------
% not actually included in most theses
% uncomment below lines if you want to include one
%\section*{Lay Abstract}
%\addchaptertocentry{Lay Abstract}
%\clearpage
%----------------------------------------------------------------------------------------
%	ABSTRACT PAGE
%----------------------------------------------------------------------------------------

% These commented lines produce a much nicer looking abstract, but it not how Grad Studies wants it. Though you may be able to get away with it.
%\begin{abstract}
%\addchaptertocentry{\abstractname} % Add the abstract to the table of contents
%\end{abstract}

\section*{Abstract} 
\addchaptertocentry{Abstract}
The hippocampus has been the focus of memory research for decades. 
While the functional role of this structure is not fully understood, it is widely
recognized as being vital for rapid yet accurate encoding and retrieval of associative
memories. 
Since the discovery of \ac{AHN} in the \ac{DG} 
by Altman and Das in the 1960s, many theories and models have been formulated 
to explain the functional role it plays in learning and memory. 
These models postulate different ways in which new neurons are introduced 
into the \ac{DG} and their functional importance for learning and memory.  
Few, if any, previous models have incorporated the unique properties of young 
adult-born \acp{DGC} and their developmental trajectory.
In this thesis, we propose a novel computational model of the \ac{DG} that 
incorporates the developmental trajectory of these \acp{DGC}, 
including changes in synaptic plasticity, connectivity, excitability and lateral inhibition, 
using a modified version of the \ac{RBM}. 
Our results show superior performance on memory reconstruction tasks for 
both recent and distally learned items, when the unique characteristics of young 
\ac{DGC}s are taken into account. 
The unique properties of the young neurons contribute to reducing retroactive and 
proactive interference, at both short and long time scales, despite the 
reduction in pattern separation due to their hyperexcitability. 
Our replacement model is subsequently extended to support learning dependent 
regulation of neurogenesis and apoptosis, 
using a convergence based approach to network growing and pruning.
This hybrid additive and replacement model provides 
a more realistic and flexible approach to investigating the role of 
neurogenesis regulation in learning and memory. 
Finally, we incorporate the dentate gyrus 
model into a full hippocampal circuit to assess cued recall performance. 
Once again, our neurogenesis model shows decreased 
proactive and retroactive interference.

\clearpage

\section*{Preface}
\addchaptertocentry{Preface}
This thesis consists of five chapters. \Cref{chap:intro} provides a brief literature 
review and context for the experiments discussed in following chapters.
\Cref{chap:ng-paradox} presents our initial \ac{AHN} model in the \ac{DG} 
using a static neural turnover method. 
\Cref{chap:learn-dep-ng} extends our model by simulating learning dependent 
regulation of neurogenesis and apoptosis. 
Finally, \cref{chap:full-model} presents a full hippocampal model to explore the role 
of \ac{AHN} on cued recall tasks.
 
\Cref{chap:ng-paradox} has been published in a special topics edition of the 
{\em Frontiers in Systems Neuroscience} journal series\footcite{finnegan-becker-15}. 
The content from \cref{chap:ng-paradox} has been included in this thesis under the 
terms of the Creative Commons Attribution License (CC-BY 3.0). 
However, the introduction and discussion sections have been heavily 
modified in order to better fit the format of this thesis.
The source code for all experiments presented in this document have been made publicly available 
under the terms of an MIT License\footcite{thesis}.
This thesis and its corresponding defence presentation are also available under 
a Creative Commons Attribution License (CC-BY 4.0).
 
Dr. Suzanna Becker originally proposed using the \ac{RBM} as the base \ac{ANN} model for this work 
and provided input on the design of the full hippocampal 
model described in \cref{chap:full-model}.
Dr. Becker assisted in writing the original introduction and discussion 
sections for \cref{chap:ng-paradox}, and provided a review and 
edits for the remainder of this thesis. 
The base \ac{RBM} model was written in Julia\footcite{julialang} using the Boltzmann.jl 
package\footcite{boltzmannjl}. 
I was responsible for writing the remainder of the manuscript, 
as well as designing, implementing and analyzing the results from each experiment.

Support for this research was provided by an NSERC Discovery grant awarded to Dr. Suzanna Becker,
and a personal grant from Invenia Technical Computing.

%----------------------------------------------------------------------------------------
%	ABBREVIATIONS
%----------------------------------------------------------------------------------------
% many these don't use this section, as it will be declared at first use and again each chapter. Uncomment these four lines to activate
\clearpage
\section*{\Huge Acronyms}
\addchaptertocentry{Acronyms}
\printacronyms[name] % name without an option stops the header

\clearpage

\section*{Acknowledgements}
\addchaptertocentry{Acknowledgements}
%\begin{acknowledgements}
%\addchaptertocentry{\acknowledgementname} % Add the acknowledgments to the table of contents

Thank you to the faculty and staff members in the MiNDS program for imparting 
the skills and knowledge necessary to complete this thesis. 
I am particularly appreciative of Dr. Sue Becker's 
constant support, guidance and above all patience over the past two 
years. It has been a pleasure working with you. 

I am grateful for the supportive, 
friendly and stimulating work environment fostered in the 
Neurotechnology and Neuroplasticity Lab.
Sincerest thanks to Kiret Dhindsa and Craig Hutton for 
always being around to provide suggestions and feedback.

To my family and friends, thank you for your continual support and confidence in my 
abilities throughout this endeavour.

Finally, to my partner and best friend Raelene Foisy, thank you for your wisdom, kindness 
and encouragement. This would not have been possible without you.

%\end{acknowledgements}

%----------------------------------------------------------------------------------------
%	LIST OF CONTENTS/FIGURES/TABLES PAGES
%----------------------------------------------------------------------------------------

\tableofcontents % Prints the main table of contents


%----------------------------------------------------------------------------------------
%	DECLARATION PAGE
%----------------------------------------------------------------------------------------

%\begin{declaration}
%\addchaptertocentry{\authorshipname}
%
%\noindent I, \authorname, declare that this thesis titled, \enquote{\ttitle} and the work presented in it are my own. I confirm that:
%
%\begin{itemize} 
%\item List each chapter
%\item and what you have done for it
%\end{itemize}
% 
%%\noindent Signed:\\
%%\rule[0.5em]{25em}{0.5pt} % This prints a line for the signature
% 
%%\noindent Date:\\
%%\rule[0.5em]{25em}{0.5pt} % This prints a line to write the date
%\end{declaration}


%%%%%%%%%%%%%%%%%%%%%%%%%%%
%%%%%%%%%%%%%%%%%%%%%%%%%%%
% optional page stuff
%----------------------------------------------------------------------------------------
% can do physical constraints and symbols pages, see the original thesis example on overleaf if you want to include them at https://www.overleaf.com/latex/templates/template-for-a-masters-slash-doctoral-thesis/mkzrzktcbzfl#.VlPeicorpE4
%----------------------------------------------------------------------------------------

%----------------------------------------------------------------------------------------
%	QUOTATION PAGE
%----------------------------------------------------------------------------------------

%\vspace*{0.2\textheight}

%\noindent\enquote{\itshape Thanks to my solid academic training, today I can write hundreds of words on virtually any topic without possessing a shred of information, which is how I got a good job in journalism.}\bigbreak

%\hfill Dave Barry

%----------------------------------------------------------------------------------------
%	DEDICATION
%----------------------------------------------------------------------------------------

% \dedicatory{For/Dedicated to/To my\ldots} 

%%%%%%%%%%%%%%%%%%%%%%%%%%%
%%%%%%%%%%%%%%%%%%%%%%%%%%%
%%%%%%%%%%%%%%%%%%%%%%%%%%%



%----------------------------------------------------------------------------------------
% The following bit is just here to make sure we end up on a new page and get the total number of roman numeral
\label{lastoffront}
\clearpage
% make sure this command is on the last of your frontmatter pages, i.e. only this command, a \clearpage then \mainmatter
% should be fine without modification
%----------------------------------------------------------------------------------------

%----------------------------------------------------------------------------------------
%	THESIS CONTENT - CHAPTERS
%----------------------------------------------------------------------------------------

\mainmatter % Begin numeric (1,2,3...) page numbering

\pagestyle{thesis} % Return the page headers back to the "thesis" style

% Include the chapters of the thesis as separate files from the Chapters folder
% Uncomment the lines as you write the chapters

\chapter{Introduction}
\label{chap:intro}
The role of the hippocampus in memory has been a subject of endless
fascination for many decades. 
It is widely recognized that
the hippocampus is crucial for rapid, accurate encoding and retrieval of
associative memories. 
However, the neural mechanisms underlying these
complex operations are still relatively poorly understood.
In particular, despite the numerous theories and models put forward, 
many questions regarding the functional importance of \ac{AHN} 
remain unanswered.
In this thesis, we will be using an \ac{ANN} to explore the functional 
role \ac{AHN} and young \acp{DGC} play in learning and memory.

\section{Hippocampal Structure \& Function}
Located under the cerebral cortex, in the medial temporal lobe of the vertebrate brain,
the hippocampal structure consists of the \ac{EC}, \ac{DG}, and cornu 
ammonis sub-layers CA1 and CA3. 
Cortical sensory information from the perirhinal, parahippocampal, 
and prefrontal cortices enters the hippocampus via layers II and III of the \ac{EC}. 
Layer II inputs are projected onto the \ac{DG} and CA3 via the \ac{PP}, while 
information from layer III is relayed to the CA1 via the \ac{TA}. 
Information processed within 
the hippocampus is propagated from the CA1 layer back to the aforementioned 
cortical areas via deep-layer \ac{EC} neurons.
The \ac{DG} receives information from the \ac{EC} and projects onto the CA3 
pyramidal cells via mossy fibres. 
Due to high levels of feedforward and feedback 
inhibition from local interneurons and extremely low firing rates among 
\acp{DGC}, it is believed that the \ac{DG} serves 
to separate input patterns \citep{jung-mcnaughton-93,chawla-et-al-05, rolls-1987, oreilly_hippocampal_encoding_storage_and_recall, rolls-treves-1998}. 
Despite the sparse activation of \acp{DGC} 
within the \ac{DG}, evidence suggests that a single mossy fibre synapse is capable 
of activating many CA3 pyramidal cells, indicating that the \ac{DG} has a significant influence on 
CA3 memory encoding \citep{mcnaughton-morris-87,treves-rolls-92,
oreilly_hippocampal_encoding_storage_and_recall,
mcclelland-mcnaughton-oreilly-95,myers-scharfman-09}.
Along with receiving input from the \ac{EC} and \ac{DG} via the \ac{PP} and mossy fibres 
respectively, the CA3 receives input from itself via many recurrent collateral connections. 
These are thought to help form the auto-associative activity needed for memory reconstruction 
and/or temporal encoding. 
Recent evidence indicates that reciprocal connection may exist between the CA3 
and the \ac{DG}, and between the CA3 and the \ac{EC} \citep{CA3_DG_backprojections_bio}. 
Despite a significant focus on modelling the trisynaptic pathway 
(\ac{EC} $\rightarrow$ \ac{DG} $\rightarrow$ CA3 $\rightarrow$ CA1 $\rightarrow$ \ac{EC}), 
computational models utilizing these reciprocal connections have shown better pattern separation and recall 
capabilities \citep{CA3_DG_backprojections}. 
While \citet{hippocampal-circuit} provide a more thorough overview of the hippocampal circuitry,
figures ~\ref{fig:hippocampus-loc}, ~\ref{fig:hippocampus-ana} and ~\ref{fig:hippocampal_circuit} 
are provided as a visual summary of the hippocampal anatomy and circuitry. 
In particular, figure ~\ref{fig:hippocampal_circuit} covers the hippocampal 
layers modelled in this thesis.

\begin{figure}[!ht]
\centering
\begin{subfigure}[b]{.45\textwidth}
	\includegraphics[width=\textwidth]{Figs/hippocampus-location}
	\caption{Posterior and inferior cornua of left lateral ventricle exposed from the side \citep{gray-1918}.}
	\label{fig:hippocampus-loc}
\end{subfigure}
\qquad
\begin{subfigure}[b]{.45\textwidth}
	\centering
	\includegraphics[width=.7\textwidth, keepaspectratio=true]{Figs/hippocampus-anatomy2}
	\caption{
	Early drawing of neural circuitry in the hippocampal formation \citep{cajal-1909}. 
	The original arrows show the path of of excitation through the trisynaptic pathway, 
	but additional labels for the \ac{EC}, 
	\ac{DG}, CA3 and CA1 have been added for clarity.
	}
	\label{fig:hippocampus-ana}
\end{subfigure}
\caption{}
\label{fig:hippocampal-anatomy}
\end{figure}

\begin{figure}[!ht]
\begin{center}
\includegraphics[width=10cm]{Figs/hippocampal_circuit}
\end{center}
\caption{
All cortical sensory information from the 
perirhinal, parahippocampal, and prefrontal cortices enters the hippocampus via 
layers II and III of the \ac{EC}. 
The \ac{DG} receives input directly from layer II 
\ac{EC} axons via the \ac{PP}, where it is believed that the \ac{DG} performs 
pattern separation on the input through lateral inhibition. 
The CA3 layer receives input from both the \ac{EC} via the \ac{PP}, and 
the \ac{DG} via mossy fibres. 
The dense recurrent connections among the 
CA3 pyramidal neurons are thought to be involved in 
associative retrieval of memories. 
These pyramidal neurons relay 
information to the CA1 through the Schaffer collaterals. 
Along with input from the CA3 layer, the CA1 receives information 
from the \ac{EC} layer III axons via the \ac{TA}. 
All excitatory output leaves the hippocampal formation via back-projections 
through CA1 layer to the deep-layer neurons of the \ac{EC} and onto the 
aforementioned cortical areas.
}
\label{fig:hippocampal_circuit}
\end{figure}

Marr's theory of archicortex \citep{marr-71} was highly influential in setting the stage
for subsequent computational theories of hippocampal function. 
At the core of his theory was the proposal that an associative memory 
system requires an initial sparse coding stage followed by a 
subsequent processing stage that performs associative retrieval. 
While Marr's initial neural circuit proposed a 2 layer network consisting 
of an input layer and an associative layer with sparse coding, later this was revised   
into a 3 layer network consisting of an input layer, a sparse coding layer and an associative layer.

Subsequent modellers refined Marr's ideas and 
further suggested that these functions of coding and retrieval map onto the
known anatomical and physiological properties of the 
\ac{DG} and CA3 regions respectively. 
The assumption is that the sparse coding stage in Marr's model could represent 
the sparse activation of granule cells in the \ac{DG} and the 
associative layer would represent the 
CA3 pyramidal cells with their dense recurrent connections 
\citep{mcnaughton-morris-87,treves-rolls-92,oreilly_hippocampal_encoding_storage_and_recall,mcclelland-mcnaughton-oreilly-95,myers-scharfman-09}.
These models incorporate an important characteristic of the
mature \acp{DGC}: they are heavily regulated by feedback
inhibition, resulting in extremely sparse firing 
and high functional selectivity \citep{jung-mcnaughton-93,chawla-et-al-05}. 
Computer simulations demonstrate that the \ac{DG} is thereby able to improve its capacity for
storing overlapping memory traces by generating less overlapping
neural codes, a process that has come to be known as pattern separation
\citep{rolls-1987, oreilly_hippocampal_encoding_storage_and_recall, rolls-treves-1998}.
Similarly, a key component of many full hippocampal models is the many 
recurrent connections among CA3 pyramidal cells. 
Simulations of memory recall have demonstrated the importance of 
these recurrent connections for accurate memory recall and pattern completion 
\citep{mcnaughton-morris-87,treves-rolls-92,oreilly_hippocampal_encoding_storage_and_recall}.

The discovery of \ac{AHN}, first in rodents \citep{origin_of_microneurons,altman-das-67} 
and subsequently in a wide range of mammalian species 
including humans \citep{eriksson-et-al-98}, has forced theorists to reconsider
the computational functions of the \ac{DG}.
Several computational models
incorporating neurogenesis have been put forward. 
These models postulate a
range of functional roles for neurogenesis, including mitigating
interference 
\citep{chambers-potenza-hoffman-miranker-04,replacement_neurogenesis,wiskott-rasch-kempermann-06,becker-macqueen-wojtowicz-09,cuneo-quiroz-weisz-argibay-2012},
temporal association of items in
memory \citep{aimone-wiles-gage-06,aimone-wiles-gage-09} and 
clearance of remote hippocampal
memories \citep{chambers-potenza-hoffman-miranker-04,deisseroth-singla-toda-monje-palmer-malenka-04,additive_neurogenesis,weisz-argibay-2012}.
While these different theories are not necessarily incompatible with one
another, they make different predictions regarding the effect of temporal
spacing. 

When similar items are spaced closely in time, some models predict that
neurogenesis should increase pattern
integration \citep{aimone-wiles-gage-06,aimone-wiles-gage-09}.  
By the same token, the reverse
should be true of animals with reduced neurogenesis: they should exhibit
impaired pattern integration, and therefore, enhanced pattern separation for closely
spaced items. 
Thus factors that suppress 
neurogenesis such as stress and
irradiation \citep{gould-tanapat-mcewen-flugge-gross-fuchs-98,wojtowicz-06}
should impair pattern integration, resulting in 
{\em superior} abilities to distinguish similar items that are learned
within the same time period. 
That said, the opposite has been observed
empirically. 
Rodents with reduced neurogenesis are impaired at spatial
discriminations for closely spaced locations that are learned within the same
session \citep{clelland-et-al-09}, while rodents with running-induced elevated neurogenesis show enhanced
performance on spatial tests of pattern
separation \citep{creer-romberg-saksida-vanpraag-bussey-2010}.
Consistent with these data, humans who have
undergone several weeks of aerobic exercise training show superior performance
on a within-session behavioural test of pattern separation while those with
elevated stress and depression scores show a deficit on the same task \citep{dery-pilgrim-gibala-gillen-wojtowicz-macqueen-becker-13}.

When similar items are spaced widely in time, different predictions can be
made regarding the fate of the item in remote memory versus the newly learned item.
Most or all computational theories agree
that neurogenesis should facilitate the encoding of new items, protecting
against proactive interference from previously learned information.
Empirical data support this notion. 
For example, animals with intact levels of
neurogenesis are able to learn to discriminate olfactory odour pairs that
overlap with pairs learned several days ago, whereas irradiated animals with reduced
neurogenesis show greater proactive interference on this
task \citep{luu-sill-gao-becker-wojtowicz-smith-12}.   
On the other hand, opposing predictions arise regarding the influence of
neurogenesis on remote memories. 
Some theories predict that neurogenesis should promote clearance of remote memories
\citep{chambers-potenza-hoffman-miranker-04,deisseroth-singla-toda-monje-palmer-malenka-04,additive_neurogenesis,weisz-argibay-2012}. 
Other theories 
make the opposite prediction, that intact neurogenesis levels should protect
against retroactive 
interference of new learning on remote
memories \citep{replacement_neurogenesis,becker-macqueen-wojtowicz-09}. 
Consistent with the latter prediction, when animals with reduced neurogenesis
learn overlapping visual discriminations in different sessions
spaced several days apart, the more recently learned
discrimination disrupts the retrieval of the earlier memory
\citep{winocur-becker-luu-rosenzweig-wojtowicz-12}. 
These data support a role for
neurogenesis in minimizing retroactive interference between 
remote and recent memories. 
However, it is possible that neurogenesis plays dual roles in remote
memory, protecting some hippocampal memories from 
interference while causing other memories to decay. 

How is it that \ac{AHN} can contribute to improved memory and reduced interference
when similar items are learned within a single session as well as when items
are learned across temporal separations of days or weeks?  
The following thesis set out to investigate whether a single computational model of hippocampal
coding could accommodate the role played by neurogenesis across this wide range
of time scales. 
We propose that the functional properties of a heterogeneous ensemble 
of young and mature \acp{DGC} contributes to this improved memory 
and reduced interference among similar items. 
Studies have shown that the presence and 
developmental trajectories of adult-generated neurons contributes 
to the functional heterogeneity among neurons 
within the granule layer \citep{wang-et-al-00, mcavoy-et-al-15}.
As such, our model attempts to take this trajectory into 
account during learning.
In most if not all previous \ac{DG} models, these characteristics have been ignored 
\citep{replacement_neurogenesis, chambers-potenza-hoffman-miranker-04,
additive_neurogenesis, weisz-argibay-2012}. 
It is known that young
adult-generated neurons in the \ac{DG} 
are more plastic, have less lateral inhibition, have sparser connectivity and are
more broadly tuned than their mature counter-parts
\citep{enhanced_synaptic_plasticity,snyder-et-al-01,temprana-et-al-2015,determinants_of_sparse_activation,neurogenesis_dictating_the_tone,
marin-burgin-et-al-12}.
All of these may affect
how young \acp{DGC} learn in relation to the existing
networks of mature
\acp{DGC}.

Among existing computational hippocampal models, those that incorporate 
neurogenesis typically do so by
either replacing existing neurons by re-randomizing their weights 
\citep{replacement_neurogenesis,chambers-potenza-hoffman-miranker-04} 
or introducing new neurons with random weights \citep{additive_neurogenesis,weisz-argibay-2012}.
Several additional models have looked at how regulation of neurogenesis can impact learning and plasticity 
by simulating dynamically regulated neural turnover and replacement 
\citep{deisseroth-singla-toda-monje-palmer-malenka-04,apoptosis-neurogenesis-hebbian-networks,chambers-conroy-07}.
Studies by Butz and colleagues include a model of synaptogenesis, providing a framework for 
how neurogenesis regulation impacts synaptic rewiring and plasticity over varying time periods 
\citep{lehmann-et-al-05,butz-et-al-06,butz-et-al-08}.
However, none of these models have investigated how 
regulation of neurogenesis and apoptosis
contribute to learning as a continually evolving temporal process. 

\section{Computational Models}
\label{comp-model}
Many computational approaches to modelling cognition and memory exist. 
For example, some approaches view the mind as a system that operates on abstract 
symbols to form complex behaviours \citep{turing-50, searle-80}. 
These models can be used to quickly formulate theories about how high level 
cognitive operations might interact.
Alternatively, other models might focus on simulating the conductances of single neuron 
membrane potentials and voltage-gated ion channels \citep{hodgkin-huxley}. 
These models are useful when investigating the impact of changes to 
resting membrane potentials, or intra- and extra- cellular voltages.
\ac{ANN} models seek to simulate collections of neuron in order to learn an objective 
function \citep {hebb, perceptrons-62, PDP-v1, PDP-v2}. 
These are particularly useful when exploring how the organization of neural networks 
impact learning and memory.
\ac{ANN} models can also vary in their levels of abstraction.
Spike time and firing based 
models focus on simulating the temporal firing patterns among neurons, which can be useful when  
investigating the functional role of oscillatory patterns such as theta rhythms observed in EEG studies
\citep{integrate-fire-neurons}.
Alternatively, other \acp{ANN} simply learn a set of weights by applying a learning rule over 
a discrete set of independent patterns.
Unfortunately, the more biologically plausible conductance and spiking network models require more 
resources to compute, which reduces the size of the networks that can be simulated.
The perspective we have taken when selecting a base model to use is best summarized by George Box, 
who stated that "All models are wrong but some are useful" \citep{box-87}.
For our purposes, a non-spiking \ac{ANN} should provide enough detail to explore 
the functional impact of the hippocampal structure and interactions between layers while remaining relatively 
computationally inexpensive to simulate, allowing us to build networks containing 
thousands to millions of neurons.

The architecture or topology of \acp{ANN} can be summarized as 
a graph where nodes represent neurons and 
edges, typically called weights, represent synapses. 
As will be discussed below these graphs can 
be acyclic or cyclic, with unidirectional or bidirectional edges. 
\acp{ANN} learn by presenting labelled (supervised) or 
unlabelled (unsupervised) data to the network and using a learning rule 
to update the weights to better fit the observed data.
These learning rules update the weights between nodes in much 
the same way that synaptic plasticity 
modifies the synaptic strengths between neurons in 
biological neural systems.

Donald Hebb's theory of synaptic plasticity \citep{hebb} paved the way for the majority of 
modern learning rules, particularly unsupervised learning rules.
The core idea was that if a neuron consistently takes part in firing 
another neuron then some growth in one or 
both neurons is required to make that behaviour more efficient. 
Carla Schatz summarized it best with the phrase,
"Cells that fire together, wire together" \citep{neurons-wire}. 
This type of learning is often referred to as Hebbian learning 
in the \ac{ANN} literature.

\subsection{Feed Forward Neural Networks}

\begin{figure}[!h]
\centering
\begin{subfigure}[b]{.45\textwidth}
	\includegraphics[width=\textwidth]{Figs/Perceptron.png}
	\caption{Single layer perceptron \ac{ANN} with a set of inputs and a single output unit.}
	\label{fig:perceptron}
\end{subfigure}
\qquad
\begin{subfigure}[b]{.45\textwidth}
	\includegraphics[width=\textwidth]{Figs/MLP.png}
	\caption{Multilayer perceptron with a set of inputs, hidden layer and single output unit.}
	\label{fig:mlp}
\end{subfigure}
\label{fig:perceptrons}
\caption{}
\end{figure}

One of the first connectionist models developed by Frank Rosenblatt in the late 1950s was the perceptron. 
The perceptron model is what we would call a binary classifier today. 
This means that it attempts to learn an 
arbitrary function of the form $y = f(x)$, where $y$ can only be 0 or 1,
provided a large enough training set of corresponding $x$ and $y$ values.
The perceptron learns this function by iteratively updating a set of 
weights $W$ between the input vector and the single binary 
output unit, such that it minimizes the error between the observed and predicted 
values of $y$ given $x$ \citep{perceptrons-62}. 
A diagram of the network architecture is provided in figure ~\ref{fig:perceptron} 
and the calculations for estimating $f(x)$ and the weight 
updates can be seen in equations ~\ref{eq.perceptron_func} 
and ~\ref{eq.perceptron_update} respectively.

\begin{equation}
f(x) = \sum_{i} W_{i}x_{i} \label{eq.perceptron_func}
\end{equation}

\begin{equation}
\Delta W{i} = \epsilon(y_{\mathrm{data}} - y_{\mathrm{pred}}) \times x_{i} \label{eq.perceptron_update}
\end{equation}

In order to expand the perceptron to multiple layers, a method for sending the error signal back through 
multiple layers was required. 
Backpropagation \citep{backprop} is a widely used method for calculating this update 
for each layer of weights. 
The revised multilayer architecture and update rule are provided in figure 
~\ref{fig:mlp} and equation ~\ref{eq.derivative_err} respectively. 
For the full derivation of the update rule please see \citep{backprop}.

\begin{equation}
\Delta W{i,j} = -k\frac{\partial E}{\partial W_{i,j}} \label{eq.derivative_err}
\end{equation}

However, in order for the derivation of the above update rule to work a 
differentiable activation function needs to be used 
instead of the threshold function from the single layer perceptron. 
The logistic function provided in 
equation ~\ref{eq.logistic} is one of the most common activation functions that satisfy this 
requirement and is used in other \acp{ANN} we will discuss.

\begin{equation}
f(x) = \frac{1}{1 + \mathrm{exp}(-x)} \label{eq.logistic}
\end{equation}

Autoencoders are a special case of the multilayer perceptron, 
commonly used in modelling the hippocampus \citep{autoencoder-model, replacement_neurogenesis}.
Autoencoders consist of at least an input layer, a hidden layer and an output layer, 
where the input layer and the output layer are the same size.
In this case, rather than predicting an output variable $y$, 
the network is trying to find a latent representation of the input patterns such that it can encode and reconstruct them. 
This network has two key advantages relevant to hippocampal modelling. 
First, the network can be trained in an unsupervised 
fashion, meaning that we do not need any labelled data $y$, since the network is 
just encoding and decoding the input $x$.
Also, since the network is simply learning to encode and decode the training 
patterns it can be used as a simple model of 
the associative memory in the hippocampus.

Several issues arise with modelling the hippocampus as a multilayer perceptron.
First, the training of a large multilayered network  
using backpropagation from randomly initialized 
weights is often slow due to the number of differential calculations needed on each 
iteration through a training set. 
Second, passing the derivative of the error through each 
hidden layer results in most of the learning occurring in the last layer,
and little change in the earlier layers. 
This is known as the vanishing gradient problem \citep{vanishing_gradient}. 
Put another way, the learning typically gets lost in the noise, 
and converges on a very poor set of weights. 
Finally, this method is considered to be less biologically 
plausible due the requirement of non-local computations \citep{contrastive_learning}.

\subsection{Recurrent Neural Networks}

\Acp{RNN} are another class of \acp{ANN} which operate over cyclic graphs, 
unlike \acp{FFNN} which are acyclic. 
The cyclic connections within the network provide a 
mechanism for modelling temporal dynamics and sequence learning \citep{recurrent-nn-review}. 
Since a full literature review of \acp{RNN} is outside the scope of this thesis we 
will only provide a very brief overview of 
some of the common architectures relevant to hippocampal modelling, including the 
\ac{RBM} used in the remaining chapters.

The simplest method of learning in an \ac{RNN} is to reuse the backpropagation learning rule. 
In this case, we store that unit's activation at one or more previous time steps, 
and we simultaneously learn the weights from that unit to other units across all of these time steps.  
This is referred to as backpropagating through time \citep{backprop-time}. 
Unfortunately, when learning many time steps, this has the 
same vanishing gradient issue found in large multilayer perceptrons \citep{vanishing_gradient}.

Hopfield networks \citep{hopfield} are an \ac{RNN} which function as a type of 
content addressable or associative memory network. 
The network consists of a single 
layer of neurons which are all symmetrically interconnected with each other, except no 
neuron is connected to itself. 
The network learns by clamping patterns to the units and updating the weights 
according to equation ~\ref{eq.hopfieldnet_learning_rule}. 
The Hebbian nature of this learning rule implies 
that units with the same state (active or inactive) for the majority of patterns will, on average, 
learn to attract each other with positive 
weights and repel differing units with negative weights. 

\begin{equation}
\Delta W_{i,j} = \frac{1}{n} \sum\limits_{p=1}^n \epsilon_{i}^{p} \epsilon_{j}^{p} \label{eq.hopfieldnet_learning_rule}
\end{equation}
where $n$ is the number of patterns and $\epsilon_{i}^{p}$ specifies the unit state at element $i$ in pattern $p$.

The activation of individual units in the network is a thresholded sum over the activations, 
weighted by the corresponding weights of all incoming connections as 
can be seen in equation ~\ref{eq.hopfield_unit_activation}.

\begin{equation}
s_i = 
\begin{cases}
+1, & \text{if } \sum_{j} W_{i,j}s_j \geq \theta_{i}, \\ 
-1, & \text{otherwise}
\end{cases}
\label{eq.hopfield_unit_activation}
\end{equation}
where $\theta_i$ is the threshold for unit $i$, and $s_i$ and $s_j$ is the 
activation states for units $i$ and $j$ respectively.

The Hopfield network's binary threshold activation function, 
combined with appropriate assumptions about the order in which 
units' states are updated (updated sequentially in random order) 
can be shown to minimize the energy function in 
equation ~\ref{eq.hopfield_energy}. 
This can be used 
to monitor the global state of the network at each step. 
When a pattern is presented to the network, determining its initial state, as units' 
states are repeatedly updated, the network's global state converges to a stable energy 
minimum, referred to as an attractor state. 
Furthermore, the Hebbian weight-update equation creates energy minima around 
the stored training patterns, thereby stabilizing each pattern as an attractor state. 
Thus, as weights in the network 
are updated the energy value of the network will decrease. 
The minimization of free energy within the network, combined with the 
unsupervised and local learning rule, provides a more 
biologically plausible model of associative memory in the hippocampus.

\begin{equation}
E = - \frac{1}{2} \sum_{i,j} W_{i,j}s_{i}s_{j} + \sum_{i} \theta_{i}s_{i} \label{eq.hopfield_energy}
\end{equation}

Despite these advantages, the Hopfield network 
suffers from limited storage capacity. 
For a network with $n$ units the asymptotic upper bound is $2n$ in 
the general case. 
While efforts to improve the storage capacity of the Hopfield network have been made, 
other networks such as perceptrons still have better performance \citep{hopfield-memcapacity}.

The Boltzmann machine \citep{boltzmann_machine} is another 
type of \ac{RNN} which learns a set of weights so as to 
form a probabilistic, generative model of the training data. 
The network consists of a set of fully and reciprocally connected stochastic units, 
partitioned into visible and hidden units.
Weights in the network are updated based on the difference 
between the data-dependent expectations (distribution of the dataset) 
and the model's expectations.
Calculation of these expectations is intractable; however, they can be 
approximated through Gibbs sampling.
In this approach a Markov chain is run for every training pattern to 
approximate the data-dependent expectation, while another chain 
is run to approximate the model's expectation.
Unfortunately, these Markov chains still take significant time to stabilize.

The \ac{RBM} simplifies the Boltzmann machine by removing 
visible-to-visible and hidden-to-hidden connections, forming 
a bipartite graph as seen in figure ~\ref{fig:rbm}.
This makes the sampling of the data-dependent and model expectations 
more tractable.
Sampling time can be further reduced using a technique called \ac{CD}
\citep{hinton-cd,contrastive_divergence}.
The CD learning rule is provided in equation ~\ref{eq.rbm_learning_rule}. 
This equation includes the same positive and negative Hebbian learning terms 
representing the data-dependent expectation and the model's expectation. 
Brief Gibbs sampling is still used to obtain the visible and hidden 
unit states for the positive and negative
terms in the learning rule.
While figure ~\ref{fig:cd} shows a single step of Gibbs sampling, the 
visible and hidden units could be reconstructed for many steps to achieve 
a better approximation of the underlying distribution.

\begin{figure}[!h]
\centering
\begin{subfigure}[b]{0.38\textwidth}
	\includegraphics[width=\textwidth]{Figs/RBM.png}
	\caption{The \ac{RBM} with visible and hidden units connected via bidirectional weights (w).}
	\label{fig:rbm}
\end{subfigure}
\hfill
\begin{subfigure}[b]{0.52\textwidth}
	\includegraphics[width=\textwidth]{Figs/CD.png}
	\caption{The positive and negative phases of the contrastive divergence learning rule.}
	\label{fig:cd}
\end{subfigure}
\label{fig:rbms}
\caption{}
\end{figure}

\begin{equation}
\Delta W_{ij} = \epsilon((v_{i}h_{j})_{\mathrm{data}} - (v_{i}h_{j})_{\mathrm{recon}}) \label{eq.rbm_learning_rule}
\end{equation}

where $v_{\mathrm{data}}$ is the input vector and $h_{\mathrm{data}}$ is the data-driven hidden
state generated by clamping the states of the visible units to $v_{\mathrm{data}}$ 
and sampling the hidden units' states according to equation ~\ref{eq.sample_hidden}. 
$v_{\mathrm{recon}}$ is a reconstruction of the input vector generated by clamping the
states of the hidden units to the data-driven pattern $h_{\mathrm{data}}$ 
and sampling the states of the visible units according to equation ~\ref{eq.sample_visible}. 
$h_{\mathrm{recon}}$ is then created in the same way as $h_{\mathrm{data}}$, but by clamping the
visible units' states to $v_{\mathrm{recon}}$. 

\begin{equation}
\Delta a_i = \epsilon({v_i}_\mathrm{data} - {v_i}_\mathrm{recon})  \label{eq.vbias_update}
\end{equation}

\begin{equation}
\Delta b_j = \epsilon({h_j}_\mathrm{data} - {h_j}_\mathrm{recon}) \label{eq.hbias_update}
\end{equation}

In equations ~\ref{eq.sample_hidden} and ~\ref{eq.sample_visible} below 
$a_i$ and $b_i$ represent biases which provide a mechanism for shifting the output of the sigmoid activation function, 
similar to thresholds in other neural network models. 
Equations ~\ref{eq.vbias_update} and ~\ref{eq.hbias_update} show that $a$ and $b$ are 
updated using the same positive and negative terms used in updating $W$. 
Figure ~\ref{fig:cd} provides a visual representation of this 
learning procedure. 

\begin{equation}
p(v_{i}=1 | h) = \sigma (a_{i} + \sum_{j} h_{j}w_{ij}) \label{eq.sample_visible}
\end{equation} 

\begin{equation}
p(h_{j}=1 | v) = \sigma (b_{j} + \sum_{i} v_{i}w_{ij}) \label{eq.sample_hidden}
\end{equation}

We can see from equation ~\ref{eq.rbm_learning_rule}
that the positive Hebbian term 
associates data-driven input and hidden state vectors, while the negative
Hebbian term tries to  
``unlearn'' the association between the corresponding reconstructed visible and
hidden state vectors.  
Theoretically, the learning procedure should converge when its
internal reconstructions of the training patterns exactly match the
corresponding data-driven states.  
In general, an \ac{RBM} model's reconstructions of the training patterns are
obtained by alternatingly sampling hidden and visible unit states that 
are nearby data-driven states using the model's bottom-up and top-down weights respectively. 

Like the Hopfield network, the \ac{RBM} utilizes a local and unsupervised learning rule, which also 
minimizes the free energy within the network \citep{hopfield-rbm-similarities}. 
However, the presence of distinct visible and hidden 
units, along with the ability to stack \acp{RBM}, provides greater memory capacity. 
Furthermore, the ability to leave the \ac{RBM} unclamped, in a generative state, may provide a way of 
simulating imagination and dreaming along with memory reconstruction. 
It is for these reasons that the \ac{RBM} is used as the base \ac{ANN} for our model.

Before concluding, we would like to mention \ac{LSTM} networks as a type of \ac{RNN} 
that has had significant success on sequence and time series learning problems
\citep{lstm-sequence,lstm-timeseries}.
\Acp{LSTM} use the concept of a memory cell, also called an \ac{LSTM} block. 
This block feeds a set of inputs through a squashing function 
to read, write and keep gates, which control long and short term storage within the network. 
Once again, backpropagation can be used to send an error signal back 
through the memory cells.
Interestingly, by continuously feeding the error 
signal back through the gate weights within the same block, the vanishing 
gradient problem can be avoided \citep{lstm-orig}. 
While \acp{LSTM} are better designed for 
sequence learning discussed in the later chapters of this thesis, this was not our primary path of 
investigation and as such the simpler \ac{RBM} model satisfied the requirements for our base 
associative memory model.

For the remainder of this thesis we will being using the \ac{RBM} to explore 
the role of young \acp{DGC} in rapid encoding and recall within the hippocampus.
\Cref{chap:ng-paradox} presents a novel model of the \ac{DG}, which incorporates 
the developmental trajectory of adult-born \acp{DGC}. 
\Cref{chap:learn-dep-ng} 
adds a mechanism for modelling learning dependent regulation of neurogenesis and 
apoptosis. 
Finally, \cref{chap:full-model} presents a combined \ac{DG} and CA model 
in order to explore the role of young \acp{DGC} on full hippocampal encoding and recall.

      

\chapter{Neurogenesis paradoxically decreases both pattern separation and memory interference}
\label{chap:ng-paradox} 

In this chapter, we present a novel computational model of the \ac{DG} incorporating the
developmental trajectory of adult-born \acp{DGC}, using a modified version of the
\ac{RBM} to model the neural circuitry and learning
equations of \ac{DGC}. 
As discussed in \cref{chap:intro}, an \ac{RBM} is a type of neural network model consisting of 
1 layer of visible and 1 layer of hidden units, with each visible unit reciprocally connected
to each hidden unit. 
In our model, a single \ac{RBM} (not stacked RBMs) will represent 
the \ac{EC} input and \acp{DGC} with its visible and hidden units respectively.
As the model \acp{DGC} undergo development, they become
progressively less plastic, more sparse in their firing, and more densely
connected to their entorhinal inputs. 
We demonstrate how these properties can
explain the importance of adult-generated \acp{DGC} at both short and long time
scales. 

In the model described here, the maturational trajectory of adult born \acp{DGC}
will be loosely based on mouse data, for \acp{DGC} 
from the third week of maturation onward. 
It is at about age 3-4 weeks that adult born \acp{DGC} have established synaptic
afferent and efferent connections and are able to fire action potentials \citep{granule_cell_maturation}.
As compared to more mature neurons, \citet{enhanced_synaptic_plasticity} have shown that these young 
neurons have a higher input resistance, lower capacitance, lower activation threshold and 
a slower membrane time constant.
As a result, 3-4 week old \acp{DGC} can be described as being more excitable, 
while having smaller and slower 
action potentials \citep{enhanced_synaptic_plasticity,snyder-et-al-01}. 
Moreover, the young neurons are more sparsely connected to their \ac{PP} 
inputs from the \ac{EC} relative to mature
\acp{DGC} \citep{neurogenesis_dictating_the_tone}.
From weeks five through eight the young neurons undergo a
gradual decline in synaptic plasticity and are increasingly regulated by
feedback inhibition \citep{temprana-et-al-2015}.
By the eighth week, the physiological properties of the adult-generated \acp{DGC} are
largely indistinguishable from that of existing mature 
\acp{DGC} \citep{temprana-et-al-2015,neurogenesis_dictating_the_tone}.

\section{Methods}

In this section, we propose a novel approach to expressing neurogenesis in
an \ac{ANN} model of the \ac{DG}. 
While several replacement and
additive models of neurogenesis have looked at how new neurons affect
learning \citep[e.g.][]{replacement_neurogenesis, additive_neurogenesis}, 
few models have considered the full range of unique properties of \ac{AHN}
including the developmental trajectory of 
of adult-generated neurons: changes in plasticity, connectivity, excitability
and survival versus apoptosis. 
The primary contribution of this work is to provide a computational framework
within which all of these factors can be manipulated, 
differentiating the role of young versus mature \acp{DGC} in memory, and the
progression from one to the other. 
In the computational model described
here, we use the \ac{RBM} 
\citep{hinton-cd, smolensky-86, freund-haussler-92} architecture and learning
procedure. 

As discussed in \cref{chap:intro}, \acp{RBM} are a type of generative, 
associative neural network model commonly used in deep learning applications \citep[see
e.g.][]{deep_belief_nets,nair-hinton-09}. 
Our approach to expressing the neural trajectory of 
young \acp{DGC} in an \ac{RBM} is to incorporate additional constraints into the learning equation, 
such as a dynamic learning rate and sparsity penalties. 
While there are several advantages to \acp{RBM} as discussed in \cref{chap:intro},
it is important to note that the use of these constraints is
not limited to \acp{RBM} and could easily be applied to other
types of neural network models (eg. multilayer perceptrons, autoencoders, \acp{RNN}, etc).

\subsection{Sparsity}

In our simulations of neurogenesis, we take into consideration both sparse
coding and sparse connectivity. 
Sparse coding means that very few strongly activated neurons respond to a given event. 
This helps to improve pattern separation as it minimizes the probability of
overlap in the model's internal representation of highly 
similar input patterns. 
As noted in \cref{chap:intro}, extreme sparse coding is observed in
mature DG granule cells, but not in less mature adult-generated neurons.  
In our model, we simulate sparse coding by incorporating 
a sparsity cost constraint into the learning objective. 
Our sparse coding cost term is the average squared difference between each 
hidden unit's average activation and its target probability of activation \citep{nair-hinton-09}. 
By taking the derivative of this cost term with respect to the weights, we
obtain an added component to the learning equation that adjusts the weights so
as to penalize units whose activation deviates from a target level of
sparseness. 
The relative importance of this sparse coding term increases with the 
age of the neurons, to simulate the increased degree of connectivity with
inhibitory interneurons of mature DGCs.  
In the updated learning equation below, $q_j$ is the mean of our sampled hidden activation 
for hidden unit $j$ from equation 
\ref{eq.sample_hidden} and $p$ is our target activation probability.

\begin{equation}
\Delta W_{ij} = \epsilon ((v_{i}h_{j})_{\mathrm{data}} - (v_{i}h_{j})_{\mathrm{recon}}) - \mathrm{cost}(q_j - p) \label{eq.sparse_rbm}
\end{equation}

Sparse connectivity describes the level of interconnectedness between the
visible and hidden layers.  
As mentioned earlier, the degree of inter-connectivity is another property
that changes as the young \acp{DGC} mature. 

We simulate the maturational evolution of increased sparse coding and
decreased sparse connectivity as follows. 
In the case of sparse coding we vary the weight on the sparsity cost for each hidden
unit so that it is smaller for young neurons and larger for their mature
counterparts. 
To impose a sparse connectivity constraint, a binary matrix 
is used as a connectivity mask for the weight matrix. 
For young \acp{DGC}, only 30\% percent of their connections were 
randomly unmasked (non-zero), to simulate low connectivity. 
Thus, a young \ac{DGC} is initially connected to relatively few ECCs.  
As the hidden units
mature, the number of non-zero visible-to-hidden  
connections in the connectivity matrix for that hidden unit is increased
probabilistically. 
At the end of each weight update, 
the weight matrix is multiplied by this connectivity mask in order to maintain
the \`\`disconnected'' links and weights of zero. 

\subsection{Neuron Growth}

Our model makes the assumption that young neurons are more plastic, have less lateral  
inhibition (simulated via our sparse coding cost rather than lateral connections) 
and are more sparsely
connected than their mature 
counterparts, in accordance with biological data
\citep{enhanced_synaptic_plasticity, oswald-et-al-08, marin-burgin-et-al-12, wang-et-al-00}. 
For simplicity, we assume that each of  
these characteristics follows a temporal growth curve that can be described
with some permutation of the Gompertz  
function \citep{gompertz}. 
The Gompertz function has been used to model growth in a variety of applications ranging from 
modelling bacterial growth in biology to product demand in economics \citep{bacterial_growth, economic_growth}. 

\begin{equation}
g(t) = e^{-e^{-st}} \label{eq.gompertz}
\end{equation}

\begin{figure}[!h]
\begin{center}
\includegraphics[width=.4\textwidth]{Figs/gompertz}
\end{center}
\caption{ Gompertz function where $s$ is set to 5 and $t$ is between $-1$ and $1$.}
\label{fig:gompertz}
\end{figure}

The Gompertz function in equation \ref{eq.gompertz} defines a sigmoid-like
growth curve, where $t$ describes the time step and $s$ describes the shape or steepness of the function
as can be seen in figure \ref{fig:gompertz}. 
For our purposes, $t$ is bounded 
between $-1$ and $1$ and the $s$ is always set to 5.
To model young \ac{DGC} growth characteristics in the \ac{RBM},
each hidden neuron has its own set of parameters defining its current learning
rate and sparsity constraints. 
Additionally, each hidden unit has a time parameter representing its age.
At each simulated unit time interval, the age of a hidden unit is increased,
and its constraint parameters are updated as follows. 
The learning rate, which
can be thought of as a neuron's plasticity level, is defined as $1 - g(t)$
normalized to lie between 0.0025 and 0.1. Inversely, our sparsity cost can simply  
be taken from $g(t)$ and normalized to lie between 0 and our initial sparsity
cost of 0.9. 
Given these variable properties,   
the learning rule can be redefined as
\begin{equation}
\Delta W_{ij} = \epsilon_j ((v_{i}h_{j})_{\mathrm{data}} - (v_{i}h_{j})_{\mathrm{recon}})) - (\lambda_jW_{ij}) - \mathrm{cost}(q_j - p) \label{eq.growth}
\end{equation}
where the learning rate $\epsilon$, weight decay $\lambda$ and sparsity cost
terms are now each weighted by dynamically changing vectors of values rather
than static hyperparameters. 

\subsection{Neural Turnover}

It is difficult to estimate the rate at which adult-generated neurons undergo
apoptosis versus survival and maturation into adult \acp{DGC}. 
These processes are
governed by many factors 
\citep[see, e.g.,][]{apoptosis-review,why-neurons-die,cecchi-et-al-01,cameron-mckay-01} 
and are not completely understood.  
Generally, apoptosis among healthy neurons tends to be activity and age dependent, 
such that the older a neuron is, the more likely it is to undergo apoptosis, 
whereas greater involvement in neural coding protects a neuron from cell death 
\citep{why-neurons-die, cecchi-et-al-01} and a significant number of 
new \acp{DGC} survive to adulthood \citep{cameron-mckay-01}. 
Using these observations, we formulate a rule for determining whether a given
neuron will survive or undergo apoptosis 
based on its age and its contribution to learning and memory. 
To assess a unit's contribution to learning and memory, we define two terms: 
its specificity and average synaptic strength.
To assess stimulus specificity, we calculate the standard
deviation of each hidden unit's incoming weights, a quantity we refer to
hereafter as its \`\`differentiation".  
The justification is that hidden units with equal weight to all visible units
will be less effective at differentiating input patterns. 
Similarly, we calculate the average 
absolute value of the those incoming weights, to assess synaptic strength. 
Combining the differentiation and synaptic strength penalty terms, 
we are penalizing hidden units with incoming weights that are 
all very similar and close to zero. 
We rank each hidden neuron based on a weighted average of its 
synaptic strength, differentiation and age with equation ~\ref{eq.turnover}. 
Neurons within the lowest 5\% of this ranking undergo simulated apoptosis by
having their age reset to 0 and weights reset to random initial values (or
set to 0 in the case of bias weights). 

\begin{equation}
Z_i = (\alpha\mathrm{Strength}_i + \beta\mathrm{Differentiation}_i + \gamma(1 - \mathrm{Age}_i)) / (\alpha + \beta + \gamma) \label{eq.turnover}
\end{equation}
where
\begin{itemize}
\item $\mathrm{Strength}_i$ is the average of the weights from all visible units to a given hidden unit $i$. 
\item $\mathrm{Differentiation}_i$ is the standard deviation of the visible weights to hidden unit $i$
\item $\mathrm{Age}_i$ is our recorded age for the hidden unit $i$
\item $\alpha$, $\beta$ \& $\gamma$ are coefficients for modifying the
relative importance of the Strength, Differentiation and Age terms. 
For our simulations these are set to 0.2, 0.65 and 0.15 respectively. 
\end{itemize}

\subsection{Experiments}

Returning to our primary thesis in this chapter, what role does the developmental trajectory 
of young \acp{DGC} have on learning and memory in the \ac{DG}?
To investigate this we designed a set of experiments to monitor proactive and retroactive 
memory interference over short and long time scale.
This was achieved by training our models iteratively on highly similar patterns with the expectation 
that new similar patterns would be more difficult to learn (proactive interference) and 
similar distally learned patterns would be more easily forgotten (retroactive interference).
Noisy versions of 5 prototype classes were used to represent the highly similar (but different) 
patterns, intended to cause interference.
Interference was measured by using the Hamming distance between the input and the
reconstructed patterns.
We began by comparing an \ac{RBM} with and without sparse coding to confirm that the 
sparsity constraint successfully reduces both proactive and retroactive memory interference.
In Simulation 2, our neurogenesis model with and without sparse connectivity was compared 
with the base \ac{RBM} with a static sparsity constraint to observe how our development trajectory 
impacts memory interference.

All models simulated in the experiments reported here used contrastive divergence with 1 step Gibbs sampling on a 
single layer \ac{RBM} as described in \cref{chap:intro}.  
A learning rate of 0.0025 was used for all models lacking neurogenesis and a value 
between 0.0025 and 0.1 was used for all models that included neurogenesis. 
For all sparse coding models the expected probability of 
activation for each hidden unit (representing the target sparseness of mature \acp{DGC}) was set to 0.05. 
This is a very conservative constraint as previous models and empirical studies have this set 
at around an order of magnitude lower, ~0.004 or 0.4\% \citep{barnes-et-al-90,jung-mcnaughton-93}. 
All models had 200 visible units and 1000 
hidden units in order to roughly match the relative numbers of \ac{EC} and \ac{DG} neurons respectively
observed in rodents, as in previous models \citep{oreilly_hippocampal_encoding_storage_and_recall}.
For all experiments, each model was trained on mini-batches of 5 training patterns at a time, with 
1 sample from each parent class as described below. 
In order to simulate rapid one-shot learning, only 1
iteration through the training set was taken.
Similar to Orielly and McClelland \citeyearpar{oreilly_hippocampal_encoding_storage_and_recall}, we set the
expected probability of activation of each unit in the training and test patterns (representing the activation
level of each \ac{EC} input unit) to be 0.1.

Each simulated model was trained on a set of binary patterns 
representing input from the \ac{EC}. 
These patterns were randomly generated, with ten percent  
of the elements of each pattern being active (set to 1.0) and the remainder
inactive (set to 0.0).  
The patterns were created as random variations on a base
set of prototypes, so as to create patterns that had varying degrees of
similarity. Initially, five binary seed patterns were created, representing
prototype patterns from 5 different classes.
For each of these classes, 10 additional overlapping prototypes were generated
by randomly resetting 20\% percent of the  
original pattern. 
From these 55 prototypes (representing 5 classes and 11 subclasses per class), 
1200 patterns were generated and partitioned into 1000 training patterns and 200 test patterns.
Each pattern was created by 
randomly resetting another 5\% of the elements in one of the 
subclass patterns.
By generating our own dataset in this way, we were able to control the similarity between patterns 
and the subsequent levels of interference produced between training sessions.

While the training and testing scenarios varied between experiments, our evaluation of performance remained 
the same. 
As an estimate of the model's ability to recognize a given test pattern, the
test pattern was presented to the model and the Hamming distance
between the input pattern and the model's reconstruction of that test pattern was calculated.
The Hamming distance was used to measure reconstruction accuracy because of its simplicity, 
as can be seen in equation \ref{eq.hamming}. 
From there the percent match was calculated using 
equation \ref{eq.match}, where $l$ is the length of the $V_{data}$ and $V_{recon}$. 
This metric serves as an 
approximation of the formal log-likelihood cost function for the Boltzmann model; however, 
it is appropriate to use an approximation to the true cost function as there are several 
other approximations such as brief gibbs sampling and small mini-batches inherent to the \ac{RBM} model. 

\begin{equation}
D(V_{\mathrm{data}}, V_{\mathrm{recon}}) = \sum_{i=1}^{n} |(V_{\mathrm{data}_{i}} - V_{\mathrm{recon}_{i}})| \label{eq.hamming}
\end{equation}

\begin{equation}
M(V_{\mathrm{data}}, V_{\mathrm{recon}}) = 1 - ( D(V_{\mathrm{data}}, V_{\mathrm{recon}}) / l  ) \label{eq.match}
\end{equation}

In Simulation 1, we evaluated the
contribution of sparse coding (without neurogenesis) to associative 
memory in the \ac{DG} model. 
Thus, we compared 
the accuracy of the sparse coding \ac{RBM} with the base \ac{RBM} lacking a sparse
coding constraint. 
We hypothesized that the sparse coding \ac{RBM} would perform better, 
particularly for encoding highly similar patterns.
We evaluated this and all other models on both proactive and retroactive
interference. 
Learning a pattern that is highly
similar to one the model previously learned is a source of proactive
interference, potentially making it more difficult to encode the current
pattern. 
Additionally, learning the current pattern could interfere
retroactively with the model's ability to retrieve a previously learned
overlapping pattern. 
Thus each model was trained on groups of patterns, consisting 
of all training patterns from 5 of the 55 prototypes (90 patterns for a training set of 1000), 
one from each class, and immediately 
tested with the corresponding test patterns on its accuracy at reconstructing these patterns. 
As mentioned above, these patterns were presented to the model in mini-batches of 5 (1 example per class), 
and the training and test patterns had noise added to them from their prototypes by randomly resetting 5\% of the elements. 
It was then trained on another group of 90 patterns with one prototype selected from each class, 
with each successive group of 90 patterns overlapping with previously learned patterns. 
After learning the entire set of 1000 patterns consisting of 11 groups of 90, the model was 
finally tested on its ability to reconstruct all test patterns from all previously learned groups to test 
retroactive interference.

In Simulation 2, the sparsely coded \ac{RBM} with
neurogenesis, with and without sparse connectivity, was compared to the sparse
\ac{RBM}. 
We were particularly interested in how the neurogenesis model would
perform at encoding and recognizing similar patterns when they were
encountered within the same learning session versus across different learning
sessions spaced more widely in time. 
We therefore compared the performance of
the various models across 2 conditions: 
1) same-session testing in which   
the neurogenesis models had no neural turnover or growth,
2) multi-session testing which had both neural growth and neural turnover.
The same-session testing condition was created with no simulated passage of
time after training on each successive group of 90 patterns. 
For multi-session training, the passage of
time between each block of 90 patterns was simulated by 
incrementing the neuron age parameter for all hidden units. 
As discussed previously, neural growth
was simulated by incrementing the age parameter and recomputing the learning
rate and sparsity cost using the Gompertz function for each hidden unit. 
Similarly, to simulate neural turnover, we
ranked the performance of each hidden unit based on the weighted average of the synaptic strength, 
differentiation and age as described earlier, and reinitialized the lowest 5\%. 
Both neural turnover and growth were performed between sessions (or groups of 90 patterns) when we incremented 
the age parameter of the hidden units.

Our hypothesis for same-session testing was that the neurogenesis models would perform better 
than the sparsely coded \ac{RBM} without neurogenesis due to the presence of a 
few young more plastic neurons. 
Further, because the available pool of young excitable neurons would be
constant for same-session learning, making it difficult for the model to
generate distinctive traces for similar items experienced within the same
context, we predicted that sparse connectivity would be particularly
important for same-session learning.
For multi-session testing, given that a new pool of young neurons would be
available at each learning session, we hypothesized that the neurogenesis models would
perform even better than they did for same-session testing. 
Further, allowing some of the young neurons to mature and forcing less useful neurons
to be replaced was predicted to lead to improved reconstruction
accuracy with lower proactive and retroactive interference. 

\section{Results}

The results from initial tests comparing the sparse coding \ac{RBM} with the base \ac{RBM} 
show a significant improvement in overall reconstruction accuracy, as can be seen in both 
the during and post training tests shown in figures ~\ref{fig:base_figure}A and ~\ref{fig:base_figure}B 
respectively, as well in the summary graph in figure ~\ref{fig:base_figure}D.  
Similarly, the sparse coding was shown to be effectively helping to increase pattern separation, 
as can be seen by the reduced pattern overlap of the hidden unit activations in figure ~\ref{fig:base_figure}C. 
It is noteworthy that the overlap for the base \ac{RBM} was less than 30\% and the slow 
increase in performance during training suggests that it was able to learn the sparse 
representation of the dataset to some extent, but not as quickly as its sparsely constrained counterpart.

\begin{figure}[!h]
\begin{center}
\includegraphics[width=.8\textwidth]{Figs/base_figure}
\end{center}
\caption{ Simulation 1: performance of the models with and without sparse coding on 
within-session pattern reconstruction tests. 
The models were trained sequentially on 11 groups of 90 patterns, and tested on noisy 
versions of these training patterns after each group to test proactive interference and 
after all groups had completed to test retroactive interference. 
\textbf{(A)} Shows proactive interference for input reconstruction accuracies during training. 
\textbf{(B)} Shows retroactive interference for input reconstruction accuracies on each 
group after training to test retroactive interference. 
\textbf{(C)} Shows the relationship between post training reconstruction accuracy with hidden unit activation overlap. \textbf{(D)} Shows the distribution of post training accuracy over all groups.}
\label{fig:base_figure}
\end{figure}

The same session tests showed improved accuracy for both neurogenesis models, 
even without neural aging or turnover. 
This was expected since the initial ages of the hidden units were randomly selected, 
allowing the encoded characteristics of our young neurons to provide the necessary advantage. 
The sparse connectivity appears to provided a further advantage for same session testing 
as we can see in figure ~\ref{fig:samesession_figure}D. 
Interestingly, figure ~\ref{fig:samesession_figure}C shows that the neurogenesis models 
have more overlap among hidden unit activation than the normal sparse \ac{RBM}, 
which demonstrates that the neurogenesis models are providing an opportunity to 
have slightly less sparse activations due to the young neurons. 
Another interesting pattern can be seen in figure ~\ref{fig:samesession_figure}B, 
which shows a kind of recency effect found in numerous memory studies (e.g., \cite{serial_position_effect}).
At the same time, figure ~\ref{fig:samesession_figure}A shows the neurogenesis models have reduced 
proactive interference. 
The increase in accuracy on subsequent groups of patterns suggests that the neurogenesis models may be better at distinguishing novel and common elements to each group of patterns.

\begin{figure}[!h]
\begin{center}
\includegraphics[width=.8\textwidth]{Figs/samesession_figure}
\end{center}
\caption{ Simulation 2: performance of the models with and without neurogenesis and 
sparse connectivity on within-session pattern reconstruction tests. 
The models were trained sequentially on 11 groups of 90 patterns, and tested on 
noisy versions of these training patterns after each group to test proactive interference 
and after all groups had completed to test retroactive interference. 
\textbf{(A)} Shows proactive interference for input reconstruction accuracies during training. 
\textbf{(B)} Shows retroactive interference for input reconstruction accuracies on each group 
after training to test retroactive interference. 
\textbf{(C)} Shows the relationship between post training reconstruction accuracy with hidden unit activation overlap. \textbf{(D)} Shows the distribution of post training accuracy over all groups.}
\label{fig:samesession_figure}
\end{figure}

The multi session tests showed similar improvement as expected. 
Figure ~\ref{fig:multisession_figure}D once again shows the neurogenesis 
models outperforming the sparse \ac{RBM} models. 
We can also see from figure ~\ref{fig:multisession_figure}B a 
recency effect and reduced proactive interference from the neurogenesis models. 
However, the use of neural maturation and turnover in the multi session 
tests provided less benefit to overall performance than expected. 
While the non-sparsely connected neurogenesis model did see about
a 1\% increase in performance over the same session tests, 
the sparsely connected neurogenesis model saw no improvement and 
did about the same as its non-sparse counterpart. 
Interestingly, figure ~\ref{fig:multisession_figure}C shows that the 
increased overlap for the sparsely connected model is no longer 
present for our multi session tests and instead the overlap for the non-sparsely 
connected neurogenesis model has increased. 
This latter point suggests that the sparse connectivity and neural turnover 
work in equilibrium with each other depending on the learning demands required. 

\begin{figure}[!h]
\begin{center}
\includegraphics[width=.8\textwidth]{Figs/multisession_figure}
\end{center}
\caption{ Simulation 2: performance of the models with and without neurogenesis and 
sparse connectivity on across-session pattern reconstruction tests. 
The models were trained sequentially on 11 groups of 90 patterns, and 
tested on noisy versions of these training patterns after each group to test 
proactive interference and after all groups had completed to test 
retroactive interference. 
\textbf{(A)} Shows proactive interference for input reconstruction accuracies during training. 
\textbf{(B)} Shows retroactive interference for input reconstruction accuracies on each 
group after training to test retroactive interference. 
\textbf{(C)} Shows the relationship between post training reconstruction accuracy with 
hidden unit activation overlap. 
\textbf{(D)} Shows the distribution of post training accuracy over all groups.}
\label{fig:multisession_figure}
\end{figure}

\begin{table}[!h]
\centering
\resizebox{\textwidth}{!}{
\begin{tabular}{lllll}\toprule
Simulation & Models & Means & Confidence Interval & Significant\\\midrule
1 - SameSession &&&\\
& \ac{RBM} vs SparseRBM & (0.844, 0.884) & (0.03, 0.054) & *\\ 
\\
2 - SameSession &&&\\ 
& SparseRBM vs Neurogenesis & (0.883, 0.938) & (0.035, 0.057) & *\\
& SparseRBM vs Neurogenesis Sparsely Connected & (0.883, 0.938) & (0.04, 0.065) & *\\
& Neurogenesis vs Neurogenesis Sparsely Connected & (0.93, 0.938) & (0.006, 0.01) & *\\
\\
2 - MultiSession &&&\\
& SparseRBM vs Neurogenesis & (0.883, 0.934) & (0.04, 0.06) & *\\
& SparseRBM vs Neurogenesis Sparsely Connected & (0.883, 0.932) & (0.037, 0.058) & *\\
& Neurogenesis vs Neurogenesis Sparsely Connected & (0.934, 0.932) & (-0.004, 0.0) &\\
\end{tabular}
}
\caption{
Post training summary statistics for the 3 simulations. 
Mean accuracies of each pair of models and 99\% 
bootstrapped confidence intervals around the difference between means are shown;  
*s indicate statistically significant differences (those with confidence intervals which do not include 0).
The confidence intervals were generated by calculating the difference in
mean performance of pairs of models across 20 repeated simulations with
different randomly generated training and test sets.
From these 20 repeated simulations, we generated 10,000 bootstrapped resamples,
to obtain bootstrapped estimates of the distributions of the mean differences}
\label{Tab:stats}
\end{table}

In summary, the results from the neurogenesis tests showed an improvement over 
the sparse coding \ac{RBM} in all cases with and without sparse connectivity. 
That being said, while the models with sparse connectivity did show better performance on the same session 
scenario, they showed no significant improvement for multisession tests. 
This suggests that the sparse connectivity of young neurons provides improved 
performance on pattern separation and completion tasks in the short term, but 
provide little benefit for longer term applications. 
Table ~\ref{Tab:stats} shows the mean values and confidence intervals from the 
post training tests for each simulation. 

\section{Discussion}

The main goal of this study was to investigate whether the unique characteristics of young 
adult-born \acp{DGC} during their maturation period, such as increased synaptic plasticity and 
reduced lateral inhibition \citep{enhanced_synaptic_plasticity, marin-burgin-et-al-12}, 
contribute to learning novel, highly overlapping patterns.
We were particularly interested in the potential contribution of
these various properties of young neurons to interference reduction when
similar patterns are encountered at short vs. long time spacings.

Previous modelling studies have shown that the sparse coding caused by lateral inhibition within
the \ac{DG} results in improved pattern separation \citep{oreilly_hippocampal_encoding_storage_and_recall} 
which is useful for distinguishing highly similar patterns. 
We reaffirmed this in simulation 1, where we compared the reconstruction
of highly similar patterns for an \ac{RBM} with and without a sparse coding constraint. 
Similar to previous studies, we found significantly better performance 
for the \ac{RBM} using a sparse coding constraint.

Our main finding is that the models with a mixture of young and old 
neurons did not learn a neural code that maximized pattern separation, 
and yet they outperformed models with sparser, less overlapping 
codes but lacking neurogenesis. 
This may seem counter-intuitive in light of the findings of 
simulation 1: for models lacking neural turnover, those with a sparse coding constraint 
were superior. 
An alternative explanation for these results is that the degree of pattern 
separation achieved by the control model (sparsely coded \ac{RBM} lacking neurogenesis) 
was so high (less than 0.05\% pattern overlap in some cases; see 
figure ~\ref{fig:samesession_figure}C) that it would be impossible 
for models without such a sparseness constraint on the young 
neurons to achieve the same degree of pattern separation. 
However, a closer examination 
of the distribution of pattern separation scores versus model performance makes this explanation 
seem unlikely. 
The \ac{RBM} has the flexibility to learn any neural code that is optimal for pattern 
reconstruction, ranging from a sparse code to a highly distributed code. 
In fact, the sparse \ac{RBM} 
and the \ac{RBM} with neurogenesis produced codes with varying degrees of pattern separation in 
different cases (see figure ~\ref{fig:samesession_figure}C), 
and there was considerable overlap in the distributions of pattern 
separation scores for the two models. 
In cases where the sparse \ac{RBM} 
achieved the highest degree of pattern separation 
(the bottom tail of the distribution in figure ~\ref{fig:samesession_figure}C), 
the sparse \ac{RBM} actually performed most poorly. 
In other cases where the sparse \ac{RBM} converged 
to somewhat less sparse codes, performance appeared to be asymptotically approaching 
about 95\% (the top end of the distribution in figure ~\ref{fig:samesession_figure}C). 
On the other hand, models with neurogenesis achieved 
performance approaching 100\%, in spite of a wide 
range of pattern separation scores; in some situations the neurogenesis models achieved 
comparable pattern separation to the sparse \ac{RBM} but still produced superior performance. 
These results support our main conclusion that a heterogeneous model with a balance of 
mature more sparsely firing neurons and younger neurons with higher firing rates achieves 
superior pattern encoding relative to a purely sparse code. 
While our simulations suggest that the addition of younger, more hyperactive neurons strictly
leads to reduced pattern separation, McAvoy et al \citeyearpar{mcavoy-et-al-15} suggest that 
young neurons may counter this effect via potent feedback inhibition of mature granule cells. 
The latter mechanism could thus compensate for the increased activity in the young 
neuronal population by inducing greater sparsity in the mature population.
The net result of this could be a homeostatic maintenance of the overall 
activity level in the dentate gyrus \citep{mcavoy-et-al-15}.
In either case, pattern separation is obviously 
not a strict requirement for accurate neural coding. 
The more distributed code learned by 
the models with a pool of younger neurons seems to offer a good compromise between 
high pattern separation and high plasticity.

Sparse connectivity was found to be
critical when the model attempted to encode similar patterns encountered
within a single training session. 
In this case, the model would not have the
opportunity to generate a set of new neurons between encoding of one similar
pattern and the next, and it therefore had to rely on sparse connectivity of
the young neurons to generate distinct responses to similar patterns. 
Across a longer temporal separation, some of the young neurons would have matured while
there would be additional young, more plastic neurons available to encode
successive similar patterns. 
Thus, these additional properties of greater
plasticity and higher activation were more important for separating patterns
that were encountered across longer time scales. 
While these results shed light on the ways in which different features of
young neurons may contribute to memory, there are several limitations to our
models that will be addressed in the remaining chapters.

While our results are relatively robust to changes in the chosen training and 
evaluation methods, several limitations exist.
First, the values of our hyperparameters were largely selected based on 
Geoffrey Hinton's \citetitle{training_rbms} \citeyearpar{training_rbms}.
While our results are robust to minor variations in the learning rate, decay and sparsity 
parameters, we do not expect changes over an order of magnitude to yield the same results.
For example, changing the learning rates for \acp{DGC} from (0.0025-0.1) to 
(0.3-0.5) would likely not produce the same results presented here.
Second, since our experiments were explicitly designed to produce interference between 
training sessions, we would not expect to find the same results in other real-world datasets 
without appropriate preprocessing.
In particular, groups of highly similar patterns would need to be identified and organized into 
training sessions appropriately, so as to produce the same interference properties 
present in our synthetic dataset.

The current model using the \ac{RBM} requires reciprocal connectivity 
between the input and output layers, whereas the known anatomy of the dentate gyrus 
does not support this architecture; dentate granule cells do not project back to 
the \ac{EC}. 
However, in an elaborated version of this 
model that will be developed in \cref{chap:full-model} \citep{becker-hinton-SFN-abstract}, 
we incorporate the reciprocal connections between the CA3 and the \ac{DG}
\citep{CA3_DG_backprojections}, and between the CA3 and the \ac{EC}, 
thus providing a natural fit of the stacked \ac{RBM} architecture as described in \cref{chap:intro} to that of 
the hippocampal circuit. 
This full hippocampal circuit model will be required to explore the 
functional impact of young vs mature \acp{DGC} on hippocampal learning, particularly when 
investigating the performance changes on memory recall (pattern completion) and sequence replay tasks.  
Similarly, the generative characteristics of the \ac{RBM} combined with this stacked 
architecture provide a method of simulating imagination and dreaming, along with memory reconstruction.

Finally, we modelled neurogenesis and apoptosis as one operation with the simplified
replacement approach. 
In \cref{chap:learn-dep-ng}, our model will be extended to 
treat neurogenesis and apoptosis as two 
independent processes for regulating the population of \acp{DGC}. 
We propose creating a hybrid additive 
and replacement model in which neurogenesis can be up or down regulated in order to better 
investigate the role of neurogenesis in pattern separation and completion tasks over 
varying time spans. 

In summary, our results suggest that the developmental trajectory of adult-born 
\acp{DGC} may be important in explaining the role of young neurons 
in interference reduction at both short and long time scales. 
Interestingly, even though the young neurons decrease sparseness 
and pattern separation, they play a critical role in mitigating both retroactive and proactive interference. 
In order to address the limitation of the current model \cref{chap:learn-dep-ng} will expand it 
into a hybrid additive \& replacement model and \cref{chap:full-model} will explore the 
functional impact of \ac{DGC} maturation on full 
Hippocampal learning tasks.
 
\chapter{Learning Dependent Regulation of Neurogenesis and Apoptosis}
\label{chap:learn-dep-ng}

As discussed in \cref{chap:intro}, computational hippocampal models 
that incorporate neurogenesis typically do so by
either replacing existing neurons by re-randomizing their weights 
(e.g., \cite{replacement_neurogenesis,chambers-potenza-hoffman-miranker-04}) 
or by introducing new neurons with random weights (e.g., \cite{additive_neurogenesis,weisz-argibay-2012}).
Several additional models have looked at how regulation of neurogenesis can impact learning and plasticity 
by simulating dynamically regulated neural turnover and replacement
\citep{deisseroth-singla-toda-monje-palmer-malenka-04,apoptosis-neurogenesis-hebbian-networks,chambers-conroy-07}.
However, none have modelled neurogenesis and apoptosis as independent operations.
Such a model could prove 
extremely useful in exploring the results of recent 
studies examining the potential role 
of neurogenesis in human memory at both short and long time scales. 
Studies have shown that alcohol, stress \& depression, age 
and environmental enrichment all help to regulate rates of 
neurogenesis 
\citep{origin_of_microneurons, enrichment-and-activity-dependent-regulation-of-neurogenesis, 
enrichment-dependent-regulation-of-neurogenesis}
Likewise, a study by \citet{dery-goldstein-becker-15} showed that 
lower stress and depression scores 
were associated with improved item 
recognition over larger time spans (two weeks). 
While the stress \& depression scores were presumed to 
negatively correlate with neurogenesis levels, it remains unclear as to what extent 
neurogenesis contributed to performance on item 
recognition tasks \citep{dery-goldstein-becker-15}.
A model that can up and down regulate neurogenesis on 
memory encoding and cued recall tasks could be useful in testing 
these assumptions.
Furthermore, dynamic regulation of both neurogenesis and apoptosis could help
control the network size relative to changes in the input datasets; 
such a model could have benefits to \ac{ANN} and machine learning research as well.
In this chapter, we will expand our replacement neurogenesis model from \cref{chap:ng-paradox} 
into a more dynamic model by separating apoptosis and neurogenesis into separate processes,
allowing neurogenesis and apoptosis to be a up and down regulated appropriately.

It is difficult to 
estimate exact rates of apoptosis and neurogenesis 
in the dentate gyrus as many factors govern these complex processes. 
However, it is generally accepted that among healthy cells, apoptosis is 
activity and age dependent \citep{why-neurons-die, cecchi-et-al-01}.
Likewise, studies have shown that alcohol, stress \& depression, age 
and environmental enrichment all help to regulate rates of 
neurogenesis 
\citep{origin_of_microneurons, enrichment-and-activity-dependent-regulation-of-neurogenesis, 
enrichment-dependent-regulation-of-neurogenesis}. 
Given these regulator mechanisms, 
how can we cohesively model them in an \ac{ANN} so as to benefit learning? 
In this chapter, we will 
demonstrate how existing methods of hidden layer growing and 
pruning in \acp{ANN} can be leveraged to create such a 
hybrid model.

\section{Methods}

To review, \acp{ANN} learn datasets by minimizing some cost function. 
While different objective functions can be used depending on the 
type of \ac{ANN}, in all cases the cost function represents how well the 
network has been fit to the desired data. 
Similarly, the gradient of the cost function 
can be used to monitor learning in a \ac{ANN}. 
While many hyperparameters in 
an \ac{ANN} can be tuned to improve performance or find the minimum more quickly, 
changing the size of an \ac{ANN}'s hidden layer is one of the most common and effective. 
While increasing the size of a hidden layer will usually improve the model's fit to the training set, 
this has a diminishing return relative to the computational cost of running the network, 
and can even contribute to overfitting \citep{network-size-baum, network-size-denker, network-size-lecun}. 
As figure ~\ref{fig:size_vs_performance} demonstrates, the \ac{RBM} model 
used in \cref{chap:ng-paradox} has exactly this problem. 
The computational complexity increases 
at a linear rate, while the performance is only increasing 
sublinearly. 

\begin{figure}[!h]
\begin{center}
\includegraphics[width=.8\textwidth]{Figs/size_vs_performance}
\end{center}
\caption{The pseudo-likelihood score and 
computational cost relative to hidden layer size.}
\label{fig:size_vs_performance}
\end{figure}

Since the optimal hidden layer size depends on other hyperparameters as well 
as the dataset being learned, it is typically left to the network 
architect to decide what the appropriate 
hidden layer size should be. 
Unfortunately, this task is often 
tedious and time consuming for even the most experienced network 
architects. 
As a result, several automated methods 
have been proposed for determining the optimal hidden 
layer size, which can be grouped into two 
primary classes. 
The first class starts with a small hidden 
layer size and gradually adds neurons, while the second 
starts with a large network and prunes off neurons.

Network growing involves starting with a small hidden layer, 
often containing 0 or 1 neurons, and gradually adding new nodes. 
There are two common approaches to network growing, the 
cascade-correlation learning architecture \citep{cascade-correlation} 
and \ac{DNC} \citep{DNC}. 
Cascade-correlation learning uses a special kind of feedforward network where
each new neuron is trained on the network input and also receives input from all 
previously trained hidden neurons. 
Once a new hidden unit is trained, 
it is added to the network, and its input weights are frozen. 
This process is repeated
until some satisfactory error rate threshold is reached. 
This architecture has the benefit of allowing new neurons to be 
added to a network without impacting existing hidden units, 
which eliminates the need for existing network weights 
to be re-adjusted and speeds up training times \citep{cascade-correlation}.
\Ac{DNC} more intuitively trains a standard 
feedforward network with a single hidden unit, 
until the squared error converges, another 
hidden unit is added and the network is re-trained. 
Similar to cascade-correlation learning, 
this is repeated until some error rate threshold 
is reached.
\Ac{DNC} has the benefit of being more general, in that we could 
easily apply it to our \ac{RBM} model.
\Ac{DNC} also tends to lead to 
smaller network architectures because it can utilize existing weights 
when re-training on new hidden units \citep{DNC}, 
which better aligns with how new neurons impact 
existing neural connections in the \ac{DG} \citep{mcavoy-et-al-15}. 
Furthermore, by adjusting our learning rates and weight decays based on neuron age, as 
discussed in \cref{chap:ng-paradox}, we are already reducing the degree to which 
existing neural connections must change relative to the new neurons. 
More specifically, by having a higher learning rate and weight decay for 
young \acp{DGC} and lower values for mature \acp{DGC}, we can ensure 
that the re-training steps impact the new neurons more than existing ones.

Network pruning involves starting with a large hidden layer and gradually removing neurons.
While network growing attempts to add neurons until the additional neurons do not 
improve performance, pruning tries to remove unnecessary 
neurons until their removal degrades performance. 
The goal with network pruning is always 
to remove nodes with minimal negative impact on network performance. 
While more complex methods exist, 
the simplest approach is to use a metric for neural saliency, or how well
a given unit is contributing to the learning in the entire network \citep{optimal-brain-damage}. 
In \cref{chap:ng-paradox}, we used the 
magnitude of the weights, the standard deviation between weights, and 
the neural age to rank neurons by their saliency. 
Essentially, if the average 
magnitude of the weights for a given hidden unit is low, then it 
should have less impact on the output. 
Similarly, if the standard deviation between the weights is 
relatively small, this is a sign that the neuron is not differentiating inputs as well.
A nice property of this simplistic saliency metric is that we can clearly 
prioritize neurons to remove based on their stimulus specificity, 
synaptic strength and age.

While methods for automated hidden layer size selection exist, they only perform either
network growing or pruning, but not both.
In this chapter we propose a new 
method that can perform both network growing and pruning 
to model regulation of neurogenesis and apoptosis. 
This allows us to model the learning dependent regulation 
of neurogenesis and apoptosis observed in the existing literature.

\subsection{Monitoring Learning}

We can see that monitoring learning performance in both network 
growing and pruning are key to determining a 
stopping criterion.
However, in order to regulate both mechanisms 
we will need a way of dynamically adjusting 
the amount to grow and prune. 
This method will need to 
adapt to new patterns unlike existing methods.

The first step in regulating growing and pruning of the hidden layer 
will be to determine a metric for evaluating and monitoring learning 
performance. 
While it is not unreasonable to use the reconstruction 
error of the \ac{RBM} on the training dataset, the preferred method 
of monitoring learning is to calculate the pseudo-likelihood at the 
end of each epoch (iteration through the training set). 
The pseudo-likelihood in this situation is an approximation 
of how closely the representation of the dataset in the \ac{RBM} fits 
the actual training set.

\begin{equation}
E = -a'v - \sum \mathrm{log}(1 + e^{b + W'v}) \label{eq.energy}
\end{equation}
where $v$ is the input vector, $W$ are the weights, and $a$ and $b$ are biases.

\begin{equation}
\mathrm{PL}(v) = \frac{e^{-E(v_i)}}{(e^{-E(v_i)} + e^{-E(v_{i'})})} \label{eq.likelihood}
\end{equation}
where $v_i$ is the input vector and $v_{i'}$ is the same input vector, but with a 
random element $i$ flipped.

\subsection{Convergence Method}

We will be using the convergence of 
the pseudo-likelihood as a stopping condition, 
unlike the growing and pruning methods described above that 
used an arbitrary error rate.
This has two main benefits. 
First, by using convergence, we do not require any expectation 
of what the resulting pseudo-likelihood should be, 
making the method more robust to changes in the input data.
Second, the convergence calculation will give us a 
method for dynamically deciding how many neurons to add or remove 
at a given time, rather than always adding or removing a single 
neuron.

To monitor convergence, we simply use the ratio test, also referred to as the D'Alembert's criterion \citep{ratio-test}.

\begin{equation}
r = |\frac{\mathrm{PL}(v)_{n+1}}{\mathrm{PL}(v)_{n}}| \label{eq.conv}
\end{equation}

when:
\begin{itemize}
\item[] $r < 1$ | pseudo-likelihood is converging
\item[] $r = 1$ | pseudo-likelihood cannot converge anymore
\item[] $r > 1$ | pseudo-likelihood is diverging
\end{itemize}


So how does this relate to growing or pruning the hidden layer? 
If the pseudo-likelihood is still converging, adding more hidden units 
may still help. 
Conversely, if the pseudo-likelihood 
is not converging or even diverging, simply adding more hidden 
units likely will not help. 
However, pruning the existing layer 
may help by compressing the current network and 
making room for new neurons when the dataset changes. 
With these assumptions, we can formulate two simple 
calculations to give us the number of units to add and remove.

% size * (percent_change * (1.0 - cratio))
\begin{equation}
C = ||n\times\epsilon(1 - r)|| \label{eq.create}
\end{equation}

% size * (percent_change * dratio)
\begin{equation}
D = ||n\times\epsilon(r)|| \label{eq.destroy}
\end{equation}

In equations \ref{eq.create} and \ref{eq.destroy} $n$ is the hidden layer 
size, $r$ is our convergence ratio from \ref{eq.conv}, and 
$\epsilon$ is a maximum percentage with which to grow or prune the network. 

\subsection{Experiments}

In order to evaluate our convergence based method, 
we first needed to demonstrate that it was successfully able to determine an 
appropriate layer size on different static datasets. 
In our first test, we 
repeatedly trained our neurogenesis model with 
identical settings on different static datasets, 
where each dataset had the same number of observations, but 
some datasets had more classes to learn than others. 
Between each 
training session we used our convergence method to determine 
how many neurons to add or remove from the hidden layer; the model 
was then recreated with the appropriate hidden layer size. 
We expected that networks being trained on datasets with fewer classes would 
require smaller hidden layers and would plateau more quickly, 
while the RBMs being trained on the datasets with more classes would require larger 
hidden layers and plateau more slowly.

In order to demonstrate that our convergence based method could model 
learning dependent regulation of neurogenesis and apoptosis, we 
also needed to demonstrate that the hidden layer size appropriately 
changed in relation to learning demand. 
In our second experiment, we 
followed the same aforementioned training procedure, but instead of 
just training on the same dataset for the entire time, we periodically 
changed it to observe how the convergence method adapts. 
We expected 
to see the same initial pattern as in the previous test, 
but with a sudden pruning followed by growing 
when the dataset changed. 

For these experiments the models used a turnover of 10\%, a learning rate 
of 0.1 and a momentum of 0.9, with no weight decay or 
sparsity constraints. 
The higher learning rate and momentum values were used to help 
speed up training and ensure the network had fully learned the 
dataset prior to resizing.
The models had a starting hidden 
layer size of 150 and a visible layer size of 100. 
Datasets were generated by creating 5, 10 and 15 prototype patterns, which 
were then used to seed 1000 observations for training. 
The training data was repeatedly fit to the model and after each fitting 
session the dynamic hidden layer scaling described earlier was applied. 
This was performed for 100 repetitions to observe how the hidden 
layer size changed relative to the complexity of the input data.
During each fit session the training was terminated when 
either the pseudo-likelihood calculation converged or when 
training exceeded 100 epochs. 
These stopping conditions 
were chosen to constrain each training session, while providing 
enough training time to minimize noise between hidden layer resizing. 
While the same settings were used for the second experiment, 
the process was repeated on 3 different datasets, without reinitializing our 
model in between. 
These dataset changes were intended 
to represent a novel environment with increased learning demands.

\section{Results}

Our preliminary results from the static dataset test showed 
appropriate hidden layer growth relative to the complexity of the input data. 
Specifically, figure ~\ref{fig:static_regulated} shows that 
training on fewer classes results in less hidden layer growth and a quicker plateau, 
while training on a dataset with more classes takes longer to plateau and 
leads to larger hidden layer sizes. 
While this very clearly fits our initial hypothesis, 
it is interesting to note that for the first 25 repetitions the network trained with 10 classes 
required a larger hidden layer than the network trained with 15 classes.
This demonstrates that our method may still be sensitive to the learning rate, 
momentum and stopping conditions used in training the network; 
however, this should be thoroughly investigated prior to these results being finalized.

\begin{figure}[!h]
\begin{center}
\includegraphics[width=.8\textwidth]{Figs/static_regulated}
\end{center}
\caption{
Changes in size per iteration are shown for three 
different training sets with either 5, 10 or 15 different pattern classes 
per dataset for a single static dataset.}
\label{fig:static_regulated}
\end{figure}

While our preliminary results from the dynamic dataset test showed the 
sudden pruning and growth we expected, it appears that the 
apoptosis and neurogenesis are not balanced in this experiment. 
We can see in figure ~\ref{fig:dynamic_regulated} that for each new dataset introduced, 
the total number of neurons required is significantly increased. 
While in many circumstances 
this seems appropriate, it could once again suggest that our 
stopping conditions may be too strict. 

\begin{figure}[!h]
\begin{center}
\includegraphics[width=.8\textwidth]{Figs/dynamic_regulated}
\end{center}
\caption{ 
Changes in network size per iteration are shown for three 
different training sets with either 5, 10 or 15 different pattern classes 
per dataset. Over the 300 iterations the training dataset is changed 
twice (after 100 and 200 iterations) to observe how the existing network adapts 
to new data.}
\label{fig:dynamic_regulated}
\end{figure}

\section{Discussion}

Existing models have used either an additive or replacement method for introducing new 
neurons in a neurogenesis model. 
In this chapter, we proposed that a 
hybrid approach, with neurogenesis and apoptosis as 
independent operations, could provide a more biologically plausible model. 
We were particularly interested in showing whether such a method could  
1) regulate neurogenesis based on the complexity of the input data, and 
2) allow the network to adjust its rates of neurogenesis and apoptosis in response to 
dataset changes.

We began by drawing on existing literature for growing and pruning \ac{ANN} 
hidden layers. 
This revealed a common theme of using a minimum threshold 
over a cost function to determine when to add or remove neurons in the network. 
By replacing the threshold with a convergence metric 
as we approach the minimum, we were able to produce a method that can 
both grow and prune a network.

While our first experiment shows that our method is sensitive to the complexity of the input 
dataset, the relationship between input complexity and the resulting hidden 
layer size does not always hold, particularly for early repetitions. 
This may indicate that we needed 
to further tweak our hyperparameters or make the stopping criteria more flexible. 
Our second experiment showed that our method is also adaptable 
to changes in the input datasets. 
Once again, we noted that the rates of neurogenesis 
and apoptosis were not balanced, often resulting in significantly higher rates 
of neurogenesis over apoptosis when presented with new datasets. 
Again, this indicates that further tweaking of our hyperparameters or 
stopping conditions may be necessary.

While the current method simply uses the convergence ratio to determine how many neurons 
to add or remove, this can be particularly problematic in networks where the learning  
has already plateaued. 
In these cases our method would not create or destroy any neurons, despite 
being the correct choice. 
As such, our method could benefit from a stochastic offset 
parameter that could promote exploration once the network size has already converged. 
This parameter, along with the max percentage change, could be useful when examining other 
external neurogenesis factors such as exercise, depression and alcohol. 
For example, it is generally believed that exercise increases the metabolic rate, 
which can increase the number of \acp{NPC}, but the learning demand 
is what determines whether those cells are recruited or 
die off \citep{enrichment-activity-interactions}. 
In order to model this behaviour, future experiments could adjust the max percentage change 
parameter and include a convergence ratio offset to work collaboratively in much the same way.

In summary, we presented a novel approach to modelling learning dependent regulation 
of neurogenesis and apoptosis, and demonstrated how it successfully adapts to 
relative complexity and changes in the input dataset. 
Future work in this area should 
address the issue of exploration vs exploitation once the 
network has converged. 

\chapter{Neurogenesis in a full hippocampal model}
\label{chap:full-model}

As discussed in \cref{chap:ng-paradox}, a full hippocampal circuit model 
will be required to explore the functional impact of young vs mature 
\acp{DGC} on hippocampal learning, particularly when 
investigating the performance changes on memory recall (pattern completion) and 
sequence replay tasks. 
Similarly, the generative characteristics of the \ac{RBM} 
combined with this stacked architecture provide a method of 
simulating imagination and dreaming along with memory reconstruction. 
Using an existing stacked \ac{RBM} approach to represent the 
\ac{DG} and CA layers in a full hippocampal model \citep{becker-hinton-SFN-abstract, hippocampal-trbm},
we will investigate how our neurogenesis model performs on cued recall tasks.

\section{Methods}
\subsection{\acp{CRBM}}

Recall from \cref{chap:intro}, that the CA3 layer in the hippocampus has many 
recurrent collaterals which is believed to help with associative and 
temporal learning. 
While we are primarily investigating the impact of \ac{AHN} on 
cued recall tasks, any model of the CA3 will 
require a way of encoding sequences of data. 
While recurrent neural networks such as \ac{LSTM} networks 
\citep{lstm-orig} and \acp{LSM} \citep{lsm} have proven effective for 
learning such data \citep{lstm-sequence,lstm-timeseries}, 
several techniques already exist for our base \ac{RBM} model.

 \begin{figure}[!hp]
\centering
\begin{subfigure}[b]{.4\textwidth}
	\includegraphics[width=\textwidth]{Figs/TRBM}
	\caption{Restricted \ac{TRBM} diagram with feed forward hidden-to-hidden layer connections.}
	\label{fig:trbm}
\end{subfigure}
\qquad\qquad
\begin{subfigure}[b]{.35\textwidth}
	\includegraphics[width=\textwidth]{Figs/CRBM}
	\caption{\ac{CRBM} diagram with a set of autoregressive unidirectional weights (B) connect the 
		conditional visible units to the hidden layer and another set of weights (A) connect the 
		conditional visible units to the standard visible ones.
	}
	\label{fig:crbm}
\end{subfigure}
\label{fig:temporal-rbms}
\caption{}
\end{figure}

The \ac{TRBM} extends the \ac{RBM} by training a sequence of \acp{RBM}, 
one for each time step in a lookback, using feed forward visible-to-hidden 
and hidden-to-hidden connections from previous \ac{RBM} time steps \citep{trbm}. 
A common restriction on the \ac{TRBM} involves using only hidden-to-hidden 
temporal connections to speed up contrastive divergence \citep{trbm-restricted}. 
A diagram of the \ac{TRBM} architecture is provided in Figure ~\ref{fig:trbm}

The \ac{CRBM} extends the \ac{RBM} by adding visible-to-visible and visible-to-hidden
autoregressive weights from other (or conditional) visible inputs \citep{crbm-2007}.
The idea is that the \ac{RBM}'s visible units can be conditioned on other known data.
This approach has proven useful in modelling timeseries data, such as video processing, 
where the visible input can be conditioned on the same input from 
previous timesteps \citep{crbm-2007}.
That being said, the \ac{CRBM} is not limited to conditioning on these historical observations.
For example, an electricity provider may want to model the conditional dependence between 
weather and load, to better predict load requirements from weather predictions.
This could be achieved with a \ac{CRBM} by conditioning the visible load observations on 
weather forecasts for temperature, humidity, wind speed \& direction, etc.
The flexibility of our conditional inputs will prove useful later in this chapter.
A diagram of the \ac{CRBM} is in figure ~\ref{fig:crbm}. 
Subsequently, the updated learning rule is provided in 
equation ~\ref{eq.crbm_W}, and the update rules for the autoregressive 
weights $A$ and $B$ can be seen in equations ~\ref{eq.crbm_A} 
\& ~\ref{eq.crbm_B} respectively.

\begin{equation}
\Delta W_{ij} = \sum_{k} \epsilon((v_{i,t}h_{j,t})_{\mathrm{data}} - (v_{i,t}h_{j,t})_{\mathrm{recon}}) \label{eq.crbm_W}
\end{equation}

\begin{equation}
\Delta A_{k,i} = \sum_{t} \epsilon((v_{i,t}v_{k,<t})_{\mathrm{data}} - (v_{i,t}v_{k,<t})_{\mathrm{recon}}) \label{eq.crbm_A}
\end{equation}

\begin{equation}
\Delta B_{k,j} = \sum_{t} \epsilon((h_{j,t}v_{k,<t})_{\mathrm{data}} - (h_{j,t}v_{k,<t})_{\mathrm{recon}}) \label{eq.crbm_B}
\end{equation}

For our purposes, the \ac{CRBM} is the most flexible and simplest method for 
learning sequence data with little computational overhead. 
While the \ac{CRBM} 
has been an effective method of learning sequence data, it is generic enough 
that we can also use it to describe other conditional relationships. 
At the end of \cref{chap:ng-paradox} we acknowledged that 
the bidirectional weights of the \ac{RBM} are less biologically plausible, given that there 
is no evidence that the \ac{DG} has backprojections to the \ac{EC}. 
By making the \ac{DG} layer a \ac{CRBM} we can avoid this issue.
If we invert our \ac{DG} layer such that our bidirectional weights represent the mossy fibres and 
backprojections between the \ac{DG} and CA3, then we can use the autoregressive visible-to-hidden weights 
to represent \ac{EC} to \ac{DG} connections.
By doing so, the \ac{DG} will be learning patterns of activation in the CA3 by conditioning on the \ac{EC}.
This provides an \ac{RBM} based \ac{DG} model that 
correctly accounts for the directionality of the connectivity 
within the hippocampal structure.
Since we will not be simulating sequence learning in our experiments, our model will not 
be conditioning on previous timesteps.
However, this would a promising addition for future studies.

\subsection{Stacking}
Training of the multilayer model depicted in ~\ref{fig:hippocampal_model} 
begins by training the CA3 \& CA1 layer on the \ac{EC} input. 
The \ac{EC} input 
is then transformed through this layer and clamped, along with the initial \ac{EC} 
patterns, as input to the \ac{DG} layer. 
The \ac{DG} layer then learns the CA 
output conditioned on the initial \ac{EC} patterns. 
Similarly, cued recall  
testing is triggered by transforming the degraded \ac{EC} input to the 
CA3 \& CA1 layer and passing that through to the \ac{DG} layer, which 
generates a new activation 
to the CA3 \& CA1 layer. 
Finally, the CA3 \& CA1 generates the completed 
patterns from those activations.

\begin{figure}[!hp]
\begin{center}
\includegraphics[width=.4\textwidth]{Figs/hippocampal-model}
\end{center}
\caption{
As a simplification of the circuitry presented in \cref{chap:intro}, the 
\ac{DG}, modelled as a \ac{CRBM}, 
receives conditional input from the \ac{PP} and visible input from 
CA3 backprojections. 
Likewise, the bidirectional weights from the 
backprojections to represent the mossy fibres. 
The CA3 \& CA1 have been collapsed 
into 1 \ac{CRBM} with visible units representing 
input from the \ac{EC} and optional conditioning 
on previous timesteps. 
This architecture is very similar to one proposed by Becker and Hinton 
\citeyearpar{becker-hinton-SFN-abstract} using \acp{TRBM} rather than \acp{CRBM}.
}
\label{fig:hippocampal_model}
\end{figure}

\subsection{Experiments}

Returning to our primary thesis in this chapter, what role does the developmental trajectory 
of young \acp{DGC} have on full hippocampal learning and memory?
To investigate this, we used a similar approach to the one from \cref{chap:ng-paradox}.
We designed a set of experiments to monitor proactive and retroactive 
memory interference over short and long time scales.
This was achieved by training our models iteratively on highly similar patterns with the expectation 
that new similar patterns would be more difficult to learn (proactive interference) and 
distally learned similar patterns would be more easily forgotten (retroactive interference).
Noisy versions of 5 prototype classes were used to represent the highly similar sequences, 
intended to cause interference.
Unlike in \cref{chap:ng-paradox}, where the Hamming distance between the input and 
reconstruction was used to measure encoding, the distance between the source prototype 
and the reconstruction was used to measure cued recall.
If the model is able to reconstruct the prototype with a high degree of accuracy, 
despite having been trained on many variations of the prototype, we can infer 
that it has learned the general features of the training set rather than just memorizing exemplars. 
We compare our hippocampal model with and without neurogenesis to observe how the 
developmental trajectory discussed in \cref{chap:ng-paradox} impacts cued recall tasks in 
our full hippocampal model.
Our hypothesis is that our neurogenesis model will have better performance on cued recall 
tasks, with reduced proactive and retroactive memory interference across short 
and long time spans.

The models simulated in this experiment used contrastive divergence with 1 step Gibbs sampling 
on each \ac{RBM} layer in the stack. 
A learning rate of 0.0025 was used for all layers lacking neurogenesis and a value 
between 0.0025 and 0.1 was used for \ac{DG} layer of model that included neurogenesis. 
For all sparse coding models, the expected probability of 
activation for each hidden unit (representing the target sparseness of mature \acp{DGC}) was set to 0.05. 
This is a very conservative constraint as previous models and empirical studies have this set 
at around an order of magnitude lower, ~0.004 or 0.4\% \citep{barnes-et-al-90,jung-mcnaughton-93}. 
The initial network started with 200 \ac{EC} inputs, 200 CA hidden units and 1000 \ac{DG} 
hidden units in order to roughly match the relative numbers of \ac{EC}, CA and \ac{DG} neurons 
observed in rodents, as in previous models \citep{oreilly_hippocampal_encoding_storage_and_recall}.
However, for models that included neurogenesis, the \ac{DG} hidden layer was allowed to grow and shrink according 
to our regulated neurogenesis and apoptosis method described in \cref{chap:learn-dep-ng}.
Since our dataset does not directly represent temporal sequences, the 
recurrent connections in the CA layer are ignored. 
For all experiments, each model was trained on mini-batches of 5 training patterns at a time, 
with 1 sample from each parent class as described below. 
In order to simulate rapid one-shot learning, only 1
iteration through the training set was taken.
Similar to Orielly and McClelland \citeyearpar{oreilly_hippocampal_encoding_storage_and_recall}, we set the
expected probability of activation of each unit in the training and test patterns (representing the activation
level of each \ac{EC} input unit) as 0.1

Each simulated model was trained on a set of binary patterns 
representing input from the \ac{EC}. 
These patterns were randomly generated, with ten percent  
of the elements of each pattern being active (set to 1.0) and the remainder
inactive (set to 0.0).  
The patterns were created as random variations on a base
set of prototypes, so as to create patterns that had varying degrees of
similarity. Initially, five binary seed patterns were created, representing
prototype patterns from 5 different classes.
For each of these classes, 10 additional overlapping prototypes were generated
by randomly resetting 20\% percent of the  
original pattern. 
From these 55 prototypes (representing 5 classes and 11 subclasses per class), 
1200 patterns were generated and partitioned into 1000 training patterns and 200 test patterns.
Each of these patterns were created by 
randomly resetting another 5\% of the elements in one of the 
subclass patterns.

\section{Results}

The same session tests showed improved cued recall performance 
for models with neurogenesis. 
Even without neural aging or turnover, we can reduce interference in both the 
during and post training tests shown in 
Figures ~\ref{fig:samesession_recall}A and ~\ref{fig:samesession_recall}B 
respectively, as well as the summary graph in Figure ~\ref{fig:samesession_recall}D.
Again, this was expected since the initial ages of the hidden units were randomly selected, 
allowing the encoded characteristics of our young neurons to provide the necessary advantage.
Unsurprisingly, figure ~\ref{fig:samesession_recall}C shows higher 
\ac{DG} hidden unit overlap for models with neurogenesis, as the 
more active young \acp{DGC} are less selective in their firing patterns.
Interestingly, the improved performance for the neurogenesis 
models appears to be magnified relative to the single \ac{EC}-\ac{DG} 
layer network in \cref{chap:ng-paradox}.

\begin{figure}[!hp]
\begin{center}
\includegraphics[width=.8\textwidth]{Figs/samesession_recall}
\end{center}
\caption{Performance of the models 
with and without neurogenesis on within-session cued recall tests. 
The models were trained sequentially on 11 group of 90 patterns, 
and tested on noisy versions of these training patterns 
after each group to test proactive interference and after all groups had completed to test retroactive interference. 
\textbf{(A)} Proactive interference for cued recall accuracies during training. 
\textbf{(B)} Retroactive interference for cued recall accuracies on each 
group after training to test retroactive interference. 
\textbf{(C)} The relationship between post training recall accuracy with 
\ac{DG} hidden unit activation overlap. 
\textbf{(D)} The distribution of post training accuracy over all groups.
}
\label{fig:samesession_recall}
\end{figure}

The multi session tests showed similar improvement to cued recall performance. 
Once again, figure ~\ref{fig:multisession_recall}D shows the model with 
neurogenesis outperforming the model without, and 
figure ~\ref{fig:multisession_recall}B shows a 
recency effect and reduced proactive interference from the neurogenesis model. 
However, the use of neural maturation and turnover in the multi session 
tests provided less benefit to overall performance than expected. 
Again, the improved performance for the neurogenesis 
models appears to be magnified relative to the single \ac{EC}-\ac{DG} 
layer network in \cref{chap:ng-paradox}.
Interestingly, Figure ~\ref{fig:multisession_figure}C shows a further overlap in 
\ac{DG} hidden layer activation.
This is likely due to the increased population of young \acp{DGC} relative to their 
mature counter parts, using our regulated neurogenesis method from 
\cref{chap:learn-dep-ng}.

\begin{figure}[!hp]
\begin{center}
\includegraphics[width=.8\textwidth]{Figs/multisession_recall}
\end{center}
\caption{ Performance of the models 
with and without neurogenesis on across-session cued recall tests. 
The models were trained sequentially on 11 group of 90 patterns, 
and tested on noisy versions of these training patterns 
after each group to test proactive interference and after all groups had completed to test retroactive interference. 
\textbf{(A)} Proactive interference for cued recall accuracies during training. 
\textbf{(B)} Retroactive interference for cued recall accuracies on each 
group after training to test retroactive interference. 
\textbf{(C)} The relationship between post training recall accuracy with 
\ac{DG} hidden unit activation overlap. 
\textbf{(D)} The distribution of post training accuracy over all groups.
}
\label{fig:multisession_recall}
\end{figure}

\begin{table}[h]
\centering
\resizebox{\textwidth}{!}{
\begin{tabular}{lllll}\toprule
Simulation & Models & Means & Confidence Interval & Significant\\\midrule
SameSession &&&\\ 
& SparseRBM vs Neurogenesis & (0.635, 0.81) & (0.148, 0.203) & *\\
\\
MultiSession &&&\\
& SparseRBM vs Neurogenesis & (0.645, 0.811) & (0.144, 0.19) & *\\
\end{tabular}
}
\caption{
Post training summary statistics for both simulations. 
Mean accuracies of each pair of models and 99\% 
bootstrapped confidence intervals around the difference between means are shown;  *s indicate 
statistically significant differences (those with confidence intervals which do not include 0).
The confidence intervals were generated by calculating the difference in
mean performance of pairs of models across 20 repeated simulations with
different randomly generated training and test sets.
From these 20 repeated simulations, we generated 10,000 bootstrapped resamples,
to obtain bootstrapped estimates of the distributions of the mean differences}
\label{Tab:stats}
\end{table}


\section{Discussion}

In this chapter we investigated the functional impact of \ac{AHN}  
on cued recall tasks within the hippocampal structure. 
To begin, we built a full hippocampal model by stacking two \acp{CRBM}. 
The first \ac{CRBM} layer represented the CA3 \& CA1 regions by accepting 
input from the \ac{EC}.
While not utilized in our experiments, this CA3 \& CA1 layer can be conditioned on 
previous \ac{EC} input, representing the recurrent collateral connections in the CA3, 
and allowing for learning of sequence data.
The second \ac{CRBM} layer represented the \ac{DG} using the same neurogenesis 
model developed throughout this thesis, but extended to a \ac{CRBM}, which is
trained off the CA3 \& CA1 hidden layer output 
and conditioned on the \ac{EC} input. 
This modification to our neurogenesis model from the previous 
chapters addresses 1 of the 3 problems discussed at the end of \cref{chap:ng-paradox}.

While the same evaluation method from \cref{chap:ng-paradox} was used, we 
measured the Hamming distance between the reconstruction and the 
source prototype rather than the reconstruction of the input pattern itself 
in order to test cued recall.
Given that the evaluation and training methods from \cref{chap:ng-paradox} 
were largely reused, our simulation suffers the same limitations as previously discussed.
Specifically, changing the \ac{RBM} hyperparameters by 
more than an order of magnitude is likely to yield different results.
Similarly, since our experiments were explicitly designed to produce interference between 
training sessions, we would not expect to find the same results in other real-world datasets 
without appropriate preprocessing.

The primary finding from these experiments is that our neurogenesis model, 
specifically with the presence of young \acp{DGC}, helps with cued recall tasks 
in a full hippocampal model, in much the same way that we 
found they helped with rapid encoding in \cref{chap:ng-paradox}. 
Myers and Scharfman \citeyearpar{CA3_DG_backprojections} argue 
that the backprojections from the CA3 to the \ac{DG} are vital for 
learning within the \ac{DG}.
These backprojections are represented in our model by our bidirectional weights 
between the CA layer and the \ac{DG} layer, which is simply conditioned 
on \ac{EC} input. 
We believe it is these bidirectional connections which allow the young 
\acp{DGC} to interact with the full memory encoding, storage and recall cycle 
and contributing the improved cued recall performance seen in 
figures ~\ref{fig:samesession_recall} and ~\ref{fig:multisession_recall}.

While we synchronously propagated the training and test patterns 
through the CA layer to the \ac{DG}, future experiments should 
explore continuously training each layer, 
reconstructing and even generating input as asynchronous 
processes. 
In the hippocampal circuitry, the \ac{EC} sends 
information via the \ac{PP} to both the \ac{DG} and 
CA3 concurrently, which should produce a kind of race 
condition between the \ac{DG} and the CA3 layers. 
Our model simplifies this by requiring information from the \ac{EC} to 
be processed in the CA layer in order to send the teaching signal, 
via backprojections, to the \ac{DG}. 
In reality, these backprojections from the CA3 to 
the \ac{DG} are likely being activated concurrently with the mossy fibres 
from the \ac{DG} to the CA3.
This would fit with existing theories of sequence and 
reverse sequence replay within the hippocampal structure \citep{lisman-99}. 
While outside the scope of this thesis, a promising use for such a model 
would be to simulate sequence replay using real place cell, grid cell and time cell recordings 
\citep{place-cells, place-grid-cells, place-grid-models, time-cells}.

Finally, we demonstrated in this chapter that our neurogenesis model from 
\cref{chap:ng-paradox} shows the same improved performance on cued recall tasks 
in a full hippocampal circuit. 
However, we did not demonstrate what advantage the CA3 \& CA1 layer provides 
in memory encoding and recall. 
Does it help in reducing proactive and retroactive interference 
even without the presence of neurogenesis and young \acp{DGC}?
Along with conditioning on previous input in order to model the CA3 recurrent collaterals, 
future experiments should identify the independent role the 
CA3 \& CA1 layer plays on learning and memory.

In summary, we extended our neurogenesis model, described 
in the previous chapters, to include the full hippocampal
circuit. 
We found models with neurogenesis had better cued recall performance 
than models without. 
These results indicate that \ac{AHN} in the \ac{DG} may play 
an important role in recall, as well as rapid encoding. 
Future work in this area should address the following questions: 
1) How does this model behave on 
sequence data by conditioning on previous input patterns in CA3 \& CA1 layer? 
2) What advantage does the CA3 \& CA1 layer play in memory encoding and recall tasks 
within this full hippocampal architecture?
3) How well does this model simulate real-world datasets?



\chapter{Conclusion}
\label{chap:conclusion}

In this thesis we investigated the functional impact of \ac{AHN} on 
rapid memory encoding and recall in the hippocampal structure, 
focusing on the developmental trajectory of adult-generated neurons. 
Young \acp{DGC} are more plastic, have less lateral inhibition, sparser connectivity and are
more broadly tuned than their mature counter-parts
\citep{enhanced_synaptic_plasticity,snyder-et-al-01,temprana-et-al-2015,determinants_of_sparse_activation,neurogenesis_dictating_the_tone,
marin-burgin-et-al-12}.
However, it is unclear what impact these unique neurophysiological properties 
have on learning and memory.
Do these neurons contribute to learning novel and highly overlapping patterns, or 
do they help in forgetting old ones?

We chose to use \ac{RBM} based methods for our \ac{DG} and full hippocampal 
models. 
As previously discussed, despite other common neural network models, 
the \ac{RBM} has several useful properties 
which require little computational overhead. 
Unlike most other types of 
\ac{ANN} models, \acp{RBM} can be stacked and trained sequentially to 
form deep multilayer networks without relying on back-propagation. 
In contrast, deep networks trained by the error back-propagation learning procedure 
\citep{lecun-85,rumelhart-hinton-williams-86}
suffer from the vanishing gradient problem \citep{vanishing_gradient}. 
Furthermore, these models are considered to be less biologically 
plausible due the requirement of non-local computations \citep{contrastive_learning}. 
While deep multilayer networks can be pretrained using 
stacked autoencoders \citep{stacked-autoencoders},
bypassing the vanishing gradient problem, the autoencoder 
still relies on backpropagation.
The \ac{RBM} has the additional advantage of forming a generative
model of the data, allowing it to generate novel input patterns from the same data
distribution that it was trained on. 
It thereby has the potential to simulate
cognitive processes such as memory reconstruction and consolidation \citep{kali-dayan-02},
as well as imagining the future and prospective memory. 
Given that our objective was to see how 
the variability in plasticity, lateral inhibition and connectivity 
among a heterogenous pool of young and mature \acp{DGC} 
impacts memory and interference, the \ac{RBM} satisfied our requirements.

We added additional constraints to the \ac{RBM} learning rule to 
simulate the unique properties of young \acp{DGC} as they mature.
The learning rule modifications that we introduced are not specific to the \ac{RBM} and could easily be
combined with other neural network learning rules. 
For example, autoencoders,  
multilayer perceptrons and recursive neural networks can all use the same variability 
in learning rate, weight decay and sparsity constraints based on the age of the neurons 
in the \ac{DG} layer. 

While our findings from \cref{chap:ng-paradox} showed that models with a mixture 
of young and old neurons did not learn a neural code 
that maximized pattern separation, they did outperform models with sparser, 
less overlapping codes, but lacking neurogenesis. 
While these results may seem counter-intuitive given that our sparse coding model 
performed better than a base \ac{RBM}, it may suggest that a heterogeneous 
model with a balance of mature more sparsely firing neurons and younger 
neurons with higher firing rates achieves superior pattern encoding 
relative to a purely sparse code. 
McAvoy et al \citeyearpar{mcavoy-et-al-15} suggest that young neurons 
may counter their increased activity via potent feedback inhibition of mature granule cells. 
The latter mechanism could thus compensate for the increased activity in the young 
neuronal population by inducing greater sparsity in the mature population.
The net result of this could be a homeostatic maintenance of the overall 
activity level in the \ac{DG} \citep{mcavoy-et-al-15}.
Furthermore, Neher et al \citeyearpar{neher-15} claim that Hebbian learning 
in the \ac{DG} does not fully support its function as a pattern separator.
In either case, pattern separation is obviously 
not a strict requirement for accurate neural coding, and the standard 
model that the \ac{DG} and \ac{AHN} only function to help with pattern separation
during memory encoding should be revisited. 
For now, we can say that the more distributed code learned by 
the models with a pool of younger neurons seems to offer a good compromise between 
high pattern separation and high plasticity.

In order to address the limitations of our replacement approach to neurogenesis, 
discussed in \cref{chap:ng-paradox}, we presented a novel method for modelling 
learning dependent regulation of neurogenesis in \cref{chap:learn-dep-ng}. 
Using changes in the pseudo-likelihood, 
a metric for monitoring learning in \acp{RBM}, the number of \acp{DGC} to add 
or remove over time was easily regulated. 
We demonstrated how this method 
adapts to the relative complexity of the dataset being learned and how introduction 
of new novel patterns can successfully trigger apoptosis and neurogenesis functions 
in order to adapt. 
While these are only preliminary results, we believe that this approach 
is both more realistic and clearly reconciles the existing additive and replacement approaches 
to modelling neurogenesis.

In order to investigate how our neurogenesis model behaves on cued recall tasks, we 
built a full hippocampal model as described in \cref{chap:intro} to include the 
CA3 \& CA1 for associative memory. 
In \cref{chap:full-model}, we showed how our \ac{EC}-\ac{DG} \ac{RBM} could be 
stacked below a CA3-CA1 layer, creating a multilayer model,
and extended it to handle this requirement. 
Interestingly, by including the CA3 sublayer, 
we were forced to extend our base \ac{RBM} network to condition on other visible inputs with the \ac{CRBM}. 
The \ac{CRBM} extension allowed us to rephrase the connections from the \ac{EC} to the \ac{DG} as 
unidirectional autoregressive connections rather than our original bidirectional ones, which better fit 
the existing biological evidence. 
This stacked architecture 
is very similar to one proposed by Becker and Hinton \citeyearpar{becker-hinton-SFN-abstract} 
which used \acp{TRBM} instead of \acp{CRBM}.
Fox and Prescott \citeyearpar{hippocampal-trbm} extended their \ac{TRBM}
model to use a learning method that resembles particle filtering.
They focused on sequence learning on a maze tasks 
but did not use a real-world dataset.
Furthermore, they used hand set \ac{EC}-\ac{DG} weights and did 
not account for the CA3-\ac{DG} backprojections or neurogenesis.
In order to test the updated neurogenesis model on 
cued recall, we re-ran the same experiment from \cref{chap:ng-paradox}.
However, instead of 
testing the accuracy of the model to reconstruct the presented patterns, we presented degraded patterns 
and asked the network to produce the original source. 
Our preliminary results showed a significant 
improvement on cued recall tasks, specifically, the 
properties identified in \cref{chap:ng-paradox} appear 
to be magnified in the stacked architecture.

While this thesis has presented a novel model of \ac{AHN} in a full hippocampal network, 
which in turn has provided several insights about how young \acp{DGC} contribute to rapid 
memory encoding and recall within the hippocampus, there are still many more avenues of 
investigation that can be taken.
The model of the young adult-born \ac{DGC} maturation presented here 
looked specifically at changes in synaptic plasticity and 
lateral inhibition during the cell's developmental trajectory; however, it does not take into account temporal 
changes in action potential kinetics \citep{enhanced_synaptic_plasticity, marin-burgin-et-al-12}. 
This temporal component would be a valuable contribution for future work, particularly when modelling 
spatio-temporal learning and sequence replay \citep{sequence_replay}.
Furthermore, while the full hippocampal model presented in \cref{chap:full-model} 
supported recurrent CA3 connections by conditioning on previous time steps, 
we did not utilize them in our experiments. 
Modelling of these recurrent CA3 connections would also be useful when simulating 
spatio-temporal learning and sequence replay \citep{sequence_replay}.
Throughout this thesis, we have noted the ability of our \ac{RBM} based model to 
simulate imagination and dreaming by running the network in an unclamped 
generative mode.
While these generative properties were not explored in our experiments, 
interesting questions arise around the impact of young \acp{DGC} on memory 
reactivation.
For example, recent studies have shown that targeted stimulation of \acp{DGC} can induce 
context-specific fear expression \citep{memory_reactivation, memory_false_activation}, but 
it remains unclear how the presence of young \acp{DGC} impact this expression.
We believe the generative properties of our full hippocampal 
neurogenesis model could provide insights on these types of interactions.
Finally, it would be interesting to see how this model performs at simulating 
real-word datasets such as place cell, grid cell and time cell recording 
\citep{place-cells, place-grid-cells, place-grid-models, time-cells}

In summary, we have developed a novel hybrid additive \& replacement 
neurogenesis model that accounts for the developmental trajectory of 
adult-born \acp{DGC}. 
Our results suggest that this developmental 
trajectory may be important in explaining the role of young neurons in 
reducing memory interference at both short and long time scales. 
Interestingly, even though the young neurons decrease sparseness and 
pattern separation, they play a critical role in mitigating both 
retroactive and proactive interference. 
Future work in this area should 
address the following important questions: 
1) How does our model perform on temporal sequence learning? 
2) How do changes in the temporal dynamics of action potentials between young and mature 
\acp{DGC} impact these results? 
3) How well does this model fit real-world recordings 
such as place and grid cell firing behaviour?
 


%----------------------------------------------------------------------------------------
%	THESIS CONTENT - APPENDICES
%----------------------------------------------------------------------------------------

%\appendix % Cue to tell LaTeX that the following "chapters" are Appendices

% Include the appendices of the thesis as separate files from the Appendices folder
% Uncomment the lines as you write the Appendices

%\input{Appendix/Supp_Chap1.tex}
%\input{Appendices/AppendixB}
%\input{Appendices/AppendixC}

%----------------------------------------------------------------------------------------
%	BIBLIOGRAPHY
%---------------------------------------------------------------------------------------
%\printbibliography
\printbibliography[heading=bibintoc]
%\bibliography{bibliography}
%----------------------------------------------------------------------------------------

\end{document}